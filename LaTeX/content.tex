%!TEX root=./root.tex

\begin{document}

\maketitle

\section*{Disclaimer and acknowledgments}
This document served as my personal course notes when I took the course in A.Y. 19/20; if I will have time, I will try to update it to reflect the more recent material seen throughout the course. 
Most of the material comes from the course lectures held by Prof. E. Rodolà, with some additions here and there.
The document is not guaranteed to be error-free, and should always be taken \textit{cum grano salis}. Please reach out to me to report errors and/or problems.
  
A special thanks to S. Antonelli, G. Attenni, A. Caciolai and S. Esposito who helped me to draw up these notes.

\tableofcontents

\chapter{Data} 
%!TEX root=../../../root.tex

Consider a linear map $T:V \to W$, a basis $v_1,\dots,v_n \in V$ and a basis $w_1,\dots,w_m\in W$.
{
The \emph{matrix} of $T$ in these bases is the $m\times n$ array of values in $\mathbb{R}$
%
\[
\mathbf{T} = \begin{pmatrix}
    T_{1,1} & \cdots & T_{1,n} \\
    \vdots & & \vdots\\
    T_{m,1} & \cdots & T_{m,n}
  \end{pmatrix}
\]
%
whose entries $T_{i,j}$ are defined by
%
\[
{\color{darkgreen}Tv_j} = {\color{red}T_{1,j}} w_1 + \cdots + {\color{red}T_{m,j}} w_m
\]
}%

{
Hence each column of $\mathbf{T}$ contains the {\color{red}linear combination coefficients} for the {\color{darkgreen}image via $T$ of a basis vector from $V$}
}

{
In other words, the matrix encodes {\color{darkgreen}how basis vectors are mapped}, and this is enough to map all other vectors in their span, since:
\[ Tv = T ( \sum_j \alpha_j v_j ) = \sum_j T(\alpha_j v_j) = \sum_j \alpha_j {\color{darkgreen}Tv_j} \]
}

The matrix is a \emph{representation} for a linear map, and it \emph{depends on the choice of bases}.
Suppose $v \in V$ is an arbitrary vector, while $v_1,\dots,v_n$ is a basis of $V$. The matrix of $v$ wrt this basis is the $n\times 1$ matrix:
%
\[
\mathbf{v} = \begin{pmatrix}
    c_1 \\
    \vdots\\
    c_n
  \end{pmatrix}
\]
%
so that
%
\[
v = c_1 v_1 + \cdots c_n v_n
\]

Once again, we see that the matrix \emph{depends on the choice of basis} for $V$


\begin{itemize}
\item \textbf{addition}:  the matrix of $S+T$ can be obtained by summing the matrices of $S$ and $T${; this only makes sense if the \emph{same bases} are used for $S$, $T$, and $S+T$}
\item \textbf{scalar multiplication}: given $\lambda\in\mathbb{R}$, the matrix for $\lambda T$ is given by $\lambda$ times the matrix of $T$
\end{itemize}

{
In fact, we have just shown that \emph{matrices form a vector space} (Q1: what is the additive identity?) {(Q2: what is the vector space dimension?)}
}

{
We call $\mathbb{R}^{m\times n}$ the vector space of all $m\times n$ matrices with values in $\mathbb{R}$
}

{
\begin{itemize}
\item \textbf{product}: the matrix for $ST$ can be computed by the \emph{matrix product} between $\mathbf{S}$ and $\mathbf{T}$; in fact, the matrix product is defined precisely to make this work

{
Q3: is matrix product commutative?
}

{
Q4: do we need the same bases for $S:U\to V$ and $T:V \to W$?
}

\end{itemize}
}



Consider a linear map $T:V \to W$, a basis $v_1,\dots,v_n \in V$ and a basis $w_1,\dots,w_m\in W$.

%\medskip
%The $k$-th column of $\mathbf{T}$ equals the matrix vector $\mathbf{v}_k$:

%figure

From the definition of matrix product, one can show that it operates on a vector matrix as expected:
\[
\mathbf{Tv} =\mathbf{w} \quad \Leftrightarrow \quad Tv=w
\]
where $\mathbf{Tv}$ is the matrix product of $\mathbf{T}$ and $\mathbf{v}$, while $Tv$ simply denotes the function evaluation $T(v)$


{
\textbf{Remember:} $\mathbf{T}, \mathbf{v}, \mathbf{w}$ must follow a coherent choice of bases in order for the above to make sense. $\mathbf{v}$ can not be expressed in basis $\color{red}(\tilde{v}_1,\dots,\tilde{v}_n)$ if $\mathbf{T}$ only knows how to map basis vectors $\color{blue}({v}_1,\dots,{v}_n)$.
%
\[
T{\color{blue}v_j} = {\color{blue}T_{1,j}} w_1 + \cdots + {\color{blue}T_{m,j}} w_m
\]
%
%\[ Tv =  \sum_j \alpha_j T{\color{red}\tilde{v}_j} \]
}


\[
\underbrace{
\begin{pmatrix}
    T_{1,1} & \cdots & T_{1,n} \\
    \vdots & & \vdots\\
    T_{m,1} & \cdots & T_{m,n}
  \end{pmatrix}}_{\mathbf{T}}
  %
  \underbrace{
  \begin{pmatrix}
    c_1 \\
    \vdots \\
    c_n
  \end{pmatrix}}_{\mathbf{c}} =
  %
   \sum_{j=1}^n c_j \hspace{-0.6cm}
   \underbrace{\begin{pmatrix}
    {\color{red}T_{1,j}}  \\
    \vdots \\
    {\color{red}T_{m,j}}
  \end{pmatrix}}_{\mathrm{Tv_j~wrt~(w_1,\dots,w_m)}}
\]
%
\smallskip

Because recall that, for bases $v_1,\dots,v_n \in V$ and $w_1,\dots,w_m\in W$:
%
\[
Tv_j = {\color{red}T_{1,j}} w_1 + \cdots + {\color{red}T_{m,j}} w_m
\]

{
We see then that vector $c=\sum_j c_j v_j$ is mapped to $Tc = \sum_j c_j Tv_j$.

In other words, matrix product is behaving as expected.
}









\chapter{Linear regression, convexity and gradients}
%!TEX root=../../../root.tex

Consider a linear map $T:V \to W$, a basis $v_1,\dots,v_n \in V$ and a basis $w_1,\dots,w_m\in W$.
{
The \emph{matrix} of $T$ in these bases is the $m\times n$ array of values in $\mathbb{R}$
%
\[
\mathbf{T} = \begin{pmatrix}
    T_{1,1} & \cdots & T_{1,n} \\
    \vdots & & \vdots\\
    T_{m,1} & \cdots & T_{m,n}
  \end{pmatrix}
\]
%
whose entries $T_{i,j}$ are defined by
%
\[
{\color{darkgreen}Tv_j} = {\color{red}T_{1,j}} w_1 + \cdots + {\color{red}T_{m,j}} w_m
\]
}%

{
Hence each column of $\mathbf{T}$ contains the {\color{red}linear combination coefficients} for the {\color{darkgreen}image via $T$ of a basis vector from $V$}
}

{
In other words, the matrix encodes {\color{darkgreen}how basis vectors are mapped}, and this is enough to map all other vectors in their span, since:
\[ Tv = T ( \sum_j \alpha_j v_j ) = \sum_j T(\alpha_j v_j) = \sum_j \alpha_j {\color{darkgreen}Tv_j} \]
}

The matrix is a \emph{representation} for a linear map, and it \emph{depends on the choice of bases}.
Suppose $v \in V$ is an arbitrary vector, while $v_1,\dots,v_n$ is a basis of $V$. The matrix of $v$ wrt this basis is the $n\times 1$ matrix:
%
\[
\mathbf{v} = \begin{pmatrix}
    c_1 \\
    \vdots\\
    c_n
  \end{pmatrix}
\]
%
so that
%
\[
v = c_1 v_1 + \cdots c_n v_n
\]

Once again, we see that the matrix \emph{depends on the choice of basis} for $V$


\begin{itemize}
\item \textbf{addition}:  the matrix of $S+T$ can be obtained by summing the matrices of $S$ and $T${; this only makes sense if the \emph{same bases} are used for $S$, $T$, and $S+T$}
\item \textbf{scalar multiplication}: given $\lambda\in\mathbb{R}$, the matrix for $\lambda T$ is given by $\lambda$ times the matrix of $T$
\end{itemize}

{
In fact, we have just shown that \emph{matrices form a vector space} (Q1: what is the additive identity?) {(Q2: what is the vector space dimension?)}
}

{
We call $\mathbb{R}^{m\times n}$ the vector space of all $m\times n$ matrices with values in $\mathbb{R}$
}

{
\begin{itemize}
\item \textbf{product}: the matrix for $ST$ can be computed by the \emph{matrix product} between $\mathbf{S}$ and $\mathbf{T}$; in fact, the matrix product is defined precisely to make this work

{
Q3: is matrix product commutative?
}

{
Q4: do we need the same bases for $S:U\to V$ and $T:V \to W$?
}

\end{itemize}
}



Consider a linear map $T:V \to W$, a basis $v_1,\dots,v_n \in V$ and a basis $w_1,\dots,w_m\in W$.

%\medskip
%The $k$-th column of $\mathbf{T}$ equals the matrix vector $\mathbf{v}_k$:

%figure

From the definition of matrix product, one can show that it operates on a vector matrix as expected:
\[
\mathbf{Tv} =\mathbf{w} \quad \Leftrightarrow \quad Tv=w
\]
where $\mathbf{Tv}$ is the matrix product of $\mathbf{T}$ and $\mathbf{v}$, while $Tv$ simply denotes the function evaluation $T(v)$


{
\textbf{Remember:} $\mathbf{T}, \mathbf{v}, \mathbf{w}$ must follow a coherent choice of bases in order for the above to make sense. $\mathbf{v}$ can not be expressed in basis $\color{red}(\tilde{v}_1,\dots,\tilde{v}_n)$ if $\mathbf{T}$ only knows how to map basis vectors $\color{blue}({v}_1,\dots,{v}_n)$.
%
\[
T{\color{blue}v_j} = {\color{blue}T_{1,j}} w_1 + \cdots + {\color{blue}T_{m,j}} w_m
\]
%
%\[ Tv =  \sum_j \alpha_j T{\color{red}\tilde{v}_j} \]
}


\[
\underbrace{
\begin{pmatrix}
    T_{1,1} & \cdots & T_{1,n} \\
    \vdots & & \vdots\\
    T_{m,1} & \cdots & T_{m,n}
  \end{pmatrix}}_{\mathbf{T}}
  %
  \underbrace{
  \begin{pmatrix}
    c_1 \\
    \vdots \\
    c_n
  \end{pmatrix}}_{\mathbf{c}} =
  %
   \sum_{j=1}^n c_j \hspace{-0.6cm}
   \underbrace{\begin{pmatrix}
    {\color{red}T_{1,j}}  \\
    \vdots \\
    {\color{red}T_{m,j}}
  \end{pmatrix}}_{\mathrm{Tv_j~wrt~(w_1,\dots,w_m)}}
\]
%
\smallskip

Because recall that, for bases $v_1,\dots,v_n \in V$ and $w_1,\dots,w_m\in W$:
%
\[
Tv_j = {\color{red}T_{1,j}} w_1 + \cdots + {\color{red}T_{m,j}} w_m
\]

{
We see then that vector $c=\sum_j c_j v_j$ is mapped to $Tc = \sum_j c_j Tv_j$.

In other words, matrix product is behaving as expected.
}









\chapter{Nonlinear models, overfitting and regularization}
%!TEX root=../../../root.tex

Consider a linear map $T:V \to W$, a basis $v_1,\dots,v_n \in V$ and a basis $w_1,\dots,w_m\in W$.
{
The \emph{matrix} of $T$ in these bases is the $m\times n$ array of values in $\mathbb{R}$
%
\[
\mathbf{T} = \begin{pmatrix}
    T_{1,1} & \cdots & T_{1,n} \\
    \vdots & & \vdots\\
    T_{m,1} & \cdots & T_{m,n}
  \end{pmatrix}
\]
%
whose entries $T_{i,j}$ are defined by
%
\[
{\color{darkgreen}Tv_j} = {\color{red}T_{1,j}} w_1 + \cdots + {\color{red}T_{m,j}} w_m
\]
}%

{
Hence each column of $\mathbf{T}$ contains the {\color{red}linear combination coefficients} for the {\color{darkgreen}image via $T$ of a basis vector from $V$}
}

{
In other words, the matrix encodes {\color{darkgreen}how basis vectors are mapped}, and this is enough to map all other vectors in their span, since:
\[ Tv = T ( \sum_j \alpha_j v_j ) = \sum_j T(\alpha_j v_j) = \sum_j \alpha_j {\color{darkgreen}Tv_j} \]
}

The matrix is a \emph{representation} for a linear map, and it \emph{depends on the choice of bases}.
Suppose $v \in V$ is an arbitrary vector, while $v_1,\dots,v_n$ is a basis of $V$. The matrix of $v$ wrt this basis is the $n\times 1$ matrix:
%
\[
\mathbf{v} = \begin{pmatrix}
    c_1 \\
    \vdots\\
    c_n
  \end{pmatrix}
\]
%
so that
%
\[
v = c_1 v_1 + \cdots c_n v_n
\]

Once again, we see that the matrix \emph{depends on the choice of basis} for $V$


\begin{itemize}
\item \textbf{addition}:  the matrix of $S+T$ can be obtained by summing the matrices of $S$ and $T${; this only makes sense if the \emph{same bases} are used for $S$, $T$, and $S+T$}
\item \textbf{scalar multiplication}: given $\lambda\in\mathbb{R}$, the matrix for $\lambda T$ is given by $\lambda$ times the matrix of $T$
\end{itemize}

{
In fact, we have just shown that \emph{matrices form a vector space} (Q1: what is the additive identity?) {(Q2: what is the vector space dimension?)}
}

{
We call $\mathbb{R}^{m\times n}$ the vector space of all $m\times n$ matrices with values in $\mathbb{R}$
}

{
\begin{itemize}
\item \textbf{product}: the matrix for $ST$ can be computed by the \emph{matrix product} between $\mathbf{S}$ and $\mathbf{T}$; in fact, the matrix product is defined precisely to make this work

{
Q3: is matrix product commutative?
}

{
Q4: do we need the same bases for $S:U\to V$ and $T:V \to W$?
}

\end{itemize}
}



Consider a linear map $T:V \to W$, a basis $v_1,\dots,v_n \in V$ and a basis $w_1,\dots,w_m\in W$.

%\medskip
%The $k$-th column of $\mathbf{T}$ equals the matrix vector $\mathbf{v}_k$:

%figure

From the definition of matrix product, one can show that it operates on a vector matrix as expected:
\[
\mathbf{Tv} =\mathbf{w} \quad \Leftrightarrow \quad Tv=w
\]
where $\mathbf{Tv}$ is the matrix product of $\mathbf{T}$ and $\mathbf{v}$, while $Tv$ simply denotes the function evaluation $T(v)$


{
\textbf{Remember:} $\mathbf{T}, \mathbf{v}, \mathbf{w}$ must follow a coherent choice of bases in order for the above to make sense. $\mathbf{v}$ can not be expressed in basis $\color{red}(\tilde{v}_1,\dots,\tilde{v}_n)$ if $\mathbf{T}$ only knows how to map basis vectors $\color{blue}({v}_1,\dots,{v}_n)$.
%
\[
T{\color{blue}v_j} = {\color{blue}T_{1,j}} w_1 + \cdots + {\color{blue}T_{m,j}} w_m
\]
%
%\[ Tv =  \sum_j \alpha_j T{\color{red}\tilde{v}_j} \]
}


\[
\underbrace{
\begin{pmatrix}
    T_{1,1} & \cdots & T_{1,n} \\
    \vdots & & \vdots\\
    T_{m,1} & \cdots & T_{m,n}
  \end{pmatrix}}_{\mathbf{T}}
  %
  \underbrace{
  \begin{pmatrix}
    c_1 \\
    \vdots \\
    c_n
  \end{pmatrix}}_{\mathbf{c}} =
  %
   \sum_{j=1}^n c_j \hspace{-0.6cm}
   \underbrace{\begin{pmatrix}
    {\color{red}T_{1,j}}  \\
    \vdots \\
    {\color{red}T_{m,j}}
  \end{pmatrix}}_{\mathrm{Tv_j~wrt~(w_1,\dots,w_m)}}
\]
%
\smallskip

Because recall that, for bases $v_1,\dots,v_n \in V$ and $w_1,\dots,w_m\in W$:
%
\[
Tv_j = {\color{red}T_{1,j}} w_1 + \cdots + {\color{red}T_{m,j}} w_m
\]

{
We see then that vector $c=\sum_j c_j v_j$ is mapped to $Tc = \sum_j c_j Tv_j$.

In other words, matrix product is behaving as expected.
}









\chapter{Stochastic gradient descent} 
%!TEX root=../../../root.tex

Consider a linear map $T:V \to W$, a basis $v_1,\dots,v_n \in V$ and a basis $w_1,\dots,w_m\in W$.
{
The \emph{matrix} of $T$ in these bases is the $m\times n$ array of values in $\mathbb{R}$
%
\[
\mathbf{T} = \begin{pmatrix}
    T_{1,1} & \cdots & T_{1,n} \\
    \vdots & & \vdots\\
    T_{m,1} & \cdots & T_{m,n}
  \end{pmatrix}
\]
%
whose entries $T_{i,j}$ are defined by
%
\[
{\color{darkgreen}Tv_j} = {\color{red}T_{1,j}} w_1 + \cdots + {\color{red}T_{m,j}} w_m
\]
}%

{
Hence each column of $\mathbf{T}$ contains the {\color{red}linear combination coefficients} for the {\color{darkgreen}image via $T$ of a basis vector from $V$}
}

{
In other words, the matrix encodes {\color{darkgreen}how basis vectors are mapped}, and this is enough to map all other vectors in their span, since:
\[ Tv = T ( \sum_j \alpha_j v_j ) = \sum_j T(\alpha_j v_j) = \sum_j \alpha_j {\color{darkgreen}Tv_j} \]
}

The matrix is a \emph{representation} for a linear map, and it \emph{depends on the choice of bases}.
Suppose $v \in V$ is an arbitrary vector, while $v_1,\dots,v_n$ is a basis of $V$. The matrix of $v$ wrt this basis is the $n\times 1$ matrix:
%
\[
\mathbf{v} = \begin{pmatrix}
    c_1 \\
    \vdots\\
    c_n
  \end{pmatrix}
\]
%
so that
%
\[
v = c_1 v_1 + \cdots c_n v_n
\]

Once again, we see that the matrix \emph{depends on the choice of basis} for $V$


\begin{itemize}
\item \textbf{addition}:  the matrix of $S+T$ can be obtained by summing the matrices of $S$ and $T${; this only makes sense if the \emph{same bases} are used for $S$, $T$, and $S+T$}
\item \textbf{scalar multiplication}: given $\lambda\in\mathbb{R}$, the matrix for $\lambda T$ is given by $\lambda$ times the matrix of $T$
\end{itemize}

{
In fact, we have just shown that \emph{matrices form a vector space} (Q1: what is the additive identity?) {(Q2: what is the vector space dimension?)}
}

{
We call $\mathbb{R}^{m\times n}$ the vector space of all $m\times n$ matrices with values in $\mathbb{R}$
}

{
\begin{itemize}
\item \textbf{product}: the matrix for $ST$ can be computed by the \emph{matrix product} between $\mathbf{S}$ and $\mathbf{T}$; in fact, the matrix product is defined precisely to make this work

{
Q3: is matrix product commutative?
}

{
Q4: do we need the same bases for $S:U\to V$ and $T:V \to W$?
}

\end{itemize}
}



Consider a linear map $T:V \to W$, a basis $v_1,\dots,v_n \in V$ and a basis $w_1,\dots,w_m\in W$.

%\medskip
%The $k$-th column of $\mathbf{T}$ equals the matrix vector $\mathbf{v}_k$:

%figure

From the definition of matrix product, one can show that it operates on a vector matrix as expected:
\[
\mathbf{Tv} =\mathbf{w} \quad \Leftrightarrow \quad Tv=w
\]
where $\mathbf{Tv}$ is the matrix product of $\mathbf{T}$ and $\mathbf{v}$, while $Tv$ simply denotes the function evaluation $T(v)$


{
\textbf{Remember:} $\mathbf{T}, \mathbf{v}, \mathbf{w}$ must follow a coherent choice of bases in order for the above to make sense. $\mathbf{v}$ can not be expressed in basis $\color{red}(\tilde{v}_1,\dots,\tilde{v}_n)$ if $\mathbf{T}$ only knows how to map basis vectors $\color{blue}({v}_1,\dots,{v}_n)$.
%
\[
T{\color{blue}v_j} = {\color{blue}T_{1,j}} w_1 + \cdots + {\color{blue}T_{m,j}} w_m
\]
%
%\[ Tv =  \sum_j \alpha_j T{\color{red}\tilde{v}_j} \]
}


\[
\underbrace{
\begin{pmatrix}
    T_{1,1} & \cdots & T_{1,n} \\
    \vdots & & \vdots\\
    T_{m,1} & \cdots & T_{m,n}
  \end{pmatrix}}_{\mathbf{T}}
  %
  \underbrace{
  \begin{pmatrix}
    c_1 \\
    \vdots \\
    c_n
  \end{pmatrix}}_{\mathbf{c}} =
  %
   \sum_{j=1}^n c_j \hspace{-0.6cm}
   \underbrace{\begin{pmatrix}
    {\color{red}T_{1,j}}  \\
    \vdots \\
    {\color{red}T_{m,j}}
  \end{pmatrix}}_{\mathrm{Tv_j~wrt~(w_1,\dots,w_m)}}
\]
%
\smallskip

Because recall that, for bases $v_1,\dots,v_n \in V$ and $w_1,\dots,w_m\in W$:
%
\[
Tv_j = {\color{red}T_{1,j}} w_1 + \cdots + {\color{red}T_{m,j}} w_m
\]

{
We see then that vector $c=\sum_j c_j v_j$ is mapped to $Tc = \sum_j c_j Tv_j$.

In other words, matrix product is behaving as expected.
}









\chapter{Multi-layer perceptron and back-propagation} 
%!TEX root=../../../root.tex

Consider a linear map $T:V \to W$, a basis $v_1,\dots,v_n \in V$ and a basis $w_1,\dots,w_m\in W$.
{
The \emph{matrix} of $T$ in these bases is the $m\times n$ array of values in $\mathbb{R}$
%
\[
\mathbf{T} = \begin{pmatrix}
    T_{1,1} & \cdots & T_{1,n} \\
    \vdots & & \vdots\\
    T_{m,1} & \cdots & T_{m,n}
  \end{pmatrix}
\]
%
whose entries $T_{i,j}$ are defined by
%
\[
{\color{darkgreen}Tv_j} = {\color{red}T_{1,j}} w_1 + \cdots + {\color{red}T_{m,j}} w_m
\]
}%

{
Hence each column of $\mathbf{T}$ contains the {\color{red}linear combination coefficients} for the {\color{darkgreen}image via $T$ of a basis vector from $V$}
}

{
In other words, the matrix encodes {\color{darkgreen}how basis vectors are mapped}, and this is enough to map all other vectors in their span, since:
\[ Tv = T ( \sum_j \alpha_j v_j ) = \sum_j T(\alpha_j v_j) = \sum_j \alpha_j {\color{darkgreen}Tv_j} \]
}

The matrix is a \emph{representation} for a linear map, and it \emph{depends on the choice of bases}.
Suppose $v \in V$ is an arbitrary vector, while $v_1,\dots,v_n$ is a basis of $V$. The matrix of $v$ wrt this basis is the $n\times 1$ matrix:
%
\[
\mathbf{v} = \begin{pmatrix}
    c_1 \\
    \vdots\\
    c_n
  \end{pmatrix}
\]
%
so that
%
\[
v = c_1 v_1 + \cdots c_n v_n
\]

Once again, we see that the matrix \emph{depends on the choice of basis} for $V$


\begin{itemize}
\item \textbf{addition}:  the matrix of $S+T$ can be obtained by summing the matrices of $S$ and $T${; this only makes sense if the \emph{same bases} are used for $S$, $T$, and $S+T$}
\item \textbf{scalar multiplication}: given $\lambda\in\mathbb{R}$, the matrix for $\lambda T$ is given by $\lambda$ times the matrix of $T$
\end{itemize}

{
In fact, we have just shown that \emph{matrices form a vector space} (Q1: what is the additive identity?) {(Q2: what is the vector space dimension?)}
}

{
We call $\mathbb{R}^{m\times n}$ the vector space of all $m\times n$ matrices with values in $\mathbb{R}$
}

{
\begin{itemize}
\item \textbf{product}: the matrix for $ST$ can be computed by the \emph{matrix product} between $\mathbf{S}$ and $\mathbf{T}$; in fact, the matrix product is defined precisely to make this work

{
Q3: is matrix product commutative?
}

{
Q4: do we need the same bases for $S:U\to V$ and $T:V \to W$?
}

\end{itemize}
}



Consider a linear map $T:V \to W$, a basis $v_1,\dots,v_n \in V$ and a basis $w_1,\dots,w_m\in W$.

%\medskip
%The $k$-th column of $\mathbf{T}$ equals the matrix vector $\mathbf{v}_k$:

%figure

From the definition of matrix product, one can show that it operates on a vector matrix as expected:
\[
\mathbf{Tv} =\mathbf{w} \quad \Leftrightarrow \quad Tv=w
\]
where $\mathbf{Tv}$ is the matrix product of $\mathbf{T}$ and $\mathbf{v}$, while $Tv$ simply denotes the function evaluation $T(v)$


{
\textbf{Remember:} $\mathbf{T}, \mathbf{v}, \mathbf{w}$ must follow a coherent choice of bases in order for the above to make sense. $\mathbf{v}$ can not be expressed in basis $\color{red}(\tilde{v}_1,\dots,\tilde{v}_n)$ if $\mathbf{T}$ only knows how to map basis vectors $\color{blue}({v}_1,\dots,{v}_n)$.
%
\[
T{\color{blue}v_j} = {\color{blue}T_{1,j}} w_1 + \cdots + {\color{blue}T_{m,j}} w_m
\]
%
%\[ Tv =  \sum_j \alpha_j T{\color{red}\tilde{v}_j} \]
}


\[
\underbrace{
\begin{pmatrix}
    T_{1,1} & \cdots & T_{1,n} \\
    \vdots & & \vdots\\
    T_{m,1} & \cdots & T_{m,n}
  \end{pmatrix}}_{\mathbf{T}}
  %
  \underbrace{
  \begin{pmatrix}
    c_1 \\
    \vdots \\
    c_n
  \end{pmatrix}}_{\mathbf{c}} =
  %
   \sum_{j=1}^n c_j \hspace{-0.6cm}
   \underbrace{\begin{pmatrix}
    {\color{red}T_{1,j}}  \\
    \vdots \\
    {\color{red}T_{m,j}}
  \end{pmatrix}}_{\mathrm{Tv_j~wrt~(w_1,\dots,w_m)}}
\]
%
\smallskip

Because recall that, for bases $v_1,\dots,v_n \in V$ and $w_1,\dots,w_m\in W$:
%
\[
Tv_j = {\color{red}T_{1,j}} w_1 + \cdots + {\color{red}T_{m,j}} w_m
\]

{
We see then that vector $c=\sum_j c_j v_j$ is mapped to $Tc = \sum_j c_j Tv_j$.

In other words, matrix product is behaving as expected.
}









\chapter{Convolutional neural networks} 
%!TEX root=../../../root.tex

Consider a linear map $T:V \to W$, a basis $v_1,\dots,v_n \in V$ and a basis $w_1,\dots,w_m\in W$.
{
The \emph{matrix} of $T$ in these bases is the $m\times n$ array of values in $\mathbb{R}$
%
\[
\mathbf{T} = \begin{pmatrix}
    T_{1,1} & \cdots & T_{1,n} \\
    \vdots & & \vdots\\
    T_{m,1} & \cdots & T_{m,n}
  \end{pmatrix}
\]
%
whose entries $T_{i,j}$ are defined by
%
\[
{\color{darkgreen}Tv_j} = {\color{red}T_{1,j}} w_1 + \cdots + {\color{red}T_{m,j}} w_m
\]
}%

{
Hence each column of $\mathbf{T}$ contains the {\color{red}linear combination coefficients} for the {\color{darkgreen}image via $T$ of a basis vector from $V$}
}

{
In other words, the matrix encodes {\color{darkgreen}how basis vectors are mapped}, and this is enough to map all other vectors in their span, since:
\[ Tv = T ( \sum_j \alpha_j v_j ) = \sum_j T(\alpha_j v_j) = \sum_j \alpha_j {\color{darkgreen}Tv_j} \]
}

The matrix is a \emph{representation} for a linear map, and it \emph{depends on the choice of bases}.
Suppose $v \in V$ is an arbitrary vector, while $v_1,\dots,v_n$ is a basis of $V$. The matrix of $v$ wrt this basis is the $n\times 1$ matrix:
%
\[
\mathbf{v} = \begin{pmatrix}
    c_1 \\
    \vdots\\
    c_n
  \end{pmatrix}
\]
%
so that
%
\[
v = c_1 v_1 + \cdots c_n v_n
\]

Once again, we see that the matrix \emph{depends on the choice of basis} for $V$


\begin{itemize}
\item \textbf{addition}:  the matrix of $S+T$ can be obtained by summing the matrices of $S$ and $T${; this only makes sense if the \emph{same bases} are used for $S$, $T$, and $S+T$}
\item \textbf{scalar multiplication}: given $\lambda\in\mathbb{R}$, the matrix for $\lambda T$ is given by $\lambda$ times the matrix of $T$
\end{itemize}

{
In fact, we have just shown that \emph{matrices form a vector space} (Q1: what is the additive identity?) {(Q2: what is the vector space dimension?)}
}

{
We call $\mathbb{R}^{m\times n}$ the vector space of all $m\times n$ matrices with values in $\mathbb{R}$
}

{
\begin{itemize}
\item \textbf{product}: the matrix for $ST$ can be computed by the \emph{matrix product} between $\mathbf{S}$ and $\mathbf{T}$; in fact, the matrix product is defined precisely to make this work

{
Q3: is matrix product commutative?
}

{
Q4: do we need the same bases for $S:U\to V$ and $T:V \to W$?
}

\end{itemize}
}



Consider a linear map $T:V \to W$, a basis $v_1,\dots,v_n \in V$ and a basis $w_1,\dots,w_m\in W$.

%\medskip
%The $k$-th column of $\mathbf{T}$ equals the matrix vector $\mathbf{v}_k$:

%figure

From the definition of matrix product, one can show that it operates on a vector matrix as expected:
\[
\mathbf{Tv} =\mathbf{w} \quad \Leftrightarrow \quad Tv=w
\]
where $\mathbf{Tv}$ is the matrix product of $\mathbf{T}$ and $\mathbf{v}$, while $Tv$ simply denotes the function evaluation $T(v)$


{
\textbf{Remember:} $\mathbf{T}, \mathbf{v}, \mathbf{w}$ must follow a coherent choice of bases in order for the above to make sense. $\mathbf{v}$ can not be expressed in basis $\color{red}(\tilde{v}_1,\dots,\tilde{v}_n)$ if $\mathbf{T}$ only knows how to map basis vectors $\color{blue}({v}_1,\dots,{v}_n)$.
%
\[
T{\color{blue}v_j} = {\color{blue}T_{1,j}} w_1 + \cdots + {\color{blue}T_{m,j}} w_m
\]
%
%\[ Tv =  \sum_j \alpha_j T{\color{red}\tilde{v}_j} \]
}


\[
\underbrace{
\begin{pmatrix}
    T_{1,1} & \cdots & T_{1,n} \\
    \vdots & & \vdots\\
    T_{m,1} & \cdots & T_{m,n}
  \end{pmatrix}}_{\mathbf{T}}
  %
  \underbrace{
  \begin{pmatrix}
    c_1 \\
    \vdots \\
    c_n
  \end{pmatrix}}_{\mathbf{c}} =
  %
   \sum_{j=1}^n c_j \hspace{-0.6cm}
   \underbrace{\begin{pmatrix}
    {\color{red}T_{1,j}}  \\
    \vdots \\
    {\color{red}T_{m,j}}
  \end{pmatrix}}_{\mathrm{Tv_j~wrt~(w_1,\dots,w_m)}}
\]
%
\smallskip

Because recall that, for bases $v_1,\dots,v_n \in V$ and $w_1,\dots,w_m\in W$:
%
\[
Tv_j = {\color{red}T_{1,j}} w_1 + \cdots + {\color{red}T_{m,j}} w_m
\]

{
We see then that vector $c=\sum_j c_j v_j$ is mapped to $Tc = \sum_j c_j Tv_j$.

In other words, matrix product is behaving as expected.
}









\chapter{Regularization} 
%!TEX root=../../../root.tex

Consider a linear map $T:V \to W$, a basis $v_1,\dots,v_n \in V$ and a basis $w_1,\dots,w_m\in W$.
{
The \emph{matrix} of $T$ in these bases is the $m\times n$ array of values in $\mathbb{R}$
%
\[
\mathbf{T} = \begin{pmatrix}
    T_{1,1} & \cdots & T_{1,n} \\
    \vdots & & \vdots\\
    T_{m,1} & \cdots & T_{m,n}
  \end{pmatrix}
\]
%
whose entries $T_{i,j}$ are defined by
%
\[
{\color{darkgreen}Tv_j} = {\color{red}T_{1,j}} w_1 + \cdots + {\color{red}T_{m,j}} w_m
\]
}%

{
Hence each column of $\mathbf{T}$ contains the {\color{red}linear combination coefficients} for the {\color{darkgreen}image via $T$ of a basis vector from $V$}
}

{
In other words, the matrix encodes {\color{darkgreen}how basis vectors are mapped}, and this is enough to map all other vectors in their span, since:
\[ Tv = T ( \sum_j \alpha_j v_j ) = \sum_j T(\alpha_j v_j) = \sum_j \alpha_j {\color{darkgreen}Tv_j} \]
}

The matrix is a \emph{representation} for a linear map, and it \emph{depends on the choice of bases}.
Suppose $v \in V$ is an arbitrary vector, while $v_1,\dots,v_n$ is a basis of $V$. The matrix of $v$ wrt this basis is the $n\times 1$ matrix:
%
\[
\mathbf{v} = \begin{pmatrix}
    c_1 \\
    \vdots\\
    c_n
  \end{pmatrix}
\]
%
so that
%
\[
v = c_1 v_1 + \cdots c_n v_n
\]

Once again, we see that the matrix \emph{depends on the choice of basis} for $V$


\begin{itemize}
\item \textbf{addition}:  the matrix of $S+T$ can be obtained by summing the matrices of $S$ and $T${; this only makes sense if the \emph{same bases} are used for $S$, $T$, and $S+T$}
\item \textbf{scalar multiplication}: given $\lambda\in\mathbb{R}$, the matrix for $\lambda T$ is given by $\lambda$ times the matrix of $T$
\end{itemize}

{
In fact, we have just shown that \emph{matrices form a vector space} (Q1: what is the additive identity?) {(Q2: what is the vector space dimension?)}
}

{
We call $\mathbb{R}^{m\times n}$ the vector space of all $m\times n$ matrices with values in $\mathbb{R}$
}

{
\begin{itemize}
\item \textbf{product}: the matrix for $ST$ can be computed by the \emph{matrix product} between $\mathbf{S}$ and $\mathbf{T}$; in fact, the matrix product is defined precisely to make this work

{
Q3: is matrix product commutative?
}

{
Q4: do we need the same bases for $S:U\to V$ and $T:V \to W$?
}

\end{itemize}
}



Consider a linear map $T:V \to W$, a basis $v_1,\dots,v_n \in V$ and a basis $w_1,\dots,w_m\in W$.

%\medskip
%The $k$-th column of $\mathbf{T}$ equals the matrix vector $\mathbf{v}_k$:

%figure

From the definition of matrix product, one can show that it operates on a vector matrix as expected:
\[
\mathbf{Tv} =\mathbf{w} \quad \Leftrightarrow \quad Tv=w
\]
where $\mathbf{Tv}$ is the matrix product of $\mathbf{T}$ and $\mathbf{v}$, while $Tv$ simply denotes the function evaluation $T(v)$


{
\textbf{Remember:} $\mathbf{T}, \mathbf{v}, \mathbf{w}$ must follow a coherent choice of bases in order for the above to make sense. $\mathbf{v}$ can not be expressed in basis $\color{red}(\tilde{v}_1,\dots,\tilde{v}_n)$ if $\mathbf{T}$ only knows how to map basis vectors $\color{blue}({v}_1,\dots,{v}_n)$.
%
\[
T{\color{blue}v_j} = {\color{blue}T_{1,j}} w_1 + \cdots + {\color{blue}T_{m,j}} w_m
\]
%
%\[ Tv =  \sum_j \alpha_j T{\color{red}\tilde{v}_j} \]
}


\[
\underbrace{
\begin{pmatrix}
    T_{1,1} & \cdots & T_{1,n} \\
    \vdots & & \vdots\\
    T_{m,1} & \cdots & T_{m,n}
  \end{pmatrix}}_{\mathbf{T}}
  %
  \underbrace{
  \begin{pmatrix}
    c_1 \\
    \vdots \\
    c_n
  \end{pmatrix}}_{\mathbf{c}} =
  %
   \sum_{j=1}^n c_j \hspace{-0.6cm}
   \underbrace{\begin{pmatrix}
    {\color{red}T_{1,j}}  \\
    \vdots \\
    {\color{red}T_{m,j}}
  \end{pmatrix}}_{\mathrm{Tv_j~wrt~(w_1,\dots,w_m)}}
\]
%
\smallskip

Because recall that, for bases $v_1,\dots,v_n \in V$ and $w_1,\dots,w_m\in W$:
%
\[
Tv_j = {\color{red}T_{1,j}} w_1 + \cdots + {\color{red}T_{m,j}} w_m
\]

{
We see then that vector $c=\sum_j c_j v_j$ is mapped to $Tc = \sum_j c_j Tv_j$.

In other words, matrix product is behaving as expected.
}









\chapter{Deep generative models} 
%!TEX root=../../../root.tex

Consider a linear map $T:V \to W$, a basis $v_1,\dots,v_n \in V$ and a basis $w_1,\dots,w_m\in W$.
{
The \emph{matrix} of $T$ in these bases is the $m\times n$ array of values in $\mathbb{R}$
%
\[
\mathbf{T} = \begin{pmatrix}
    T_{1,1} & \cdots & T_{1,n} \\
    \vdots & & \vdots\\
    T_{m,1} & \cdots & T_{m,n}
  \end{pmatrix}
\]
%
whose entries $T_{i,j}$ are defined by
%
\[
{\color{darkgreen}Tv_j} = {\color{red}T_{1,j}} w_1 + \cdots + {\color{red}T_{m,j}} w_m
\]
}%

{
Hence each column of $\mathbf{T}$ contains the {\color{red}linear combination coefficients} for the {\color{darkgreen}image via $T$ of a basis vector from $V$}
}

{
In other words, the matrix encodes {\color{darkgreen}how basis vectors are mapped}, and this is enough to map all other vectors in their span, since:
\[ Tv = T ( \sum_j \alpha_j v_j ) = \sum_j T(\alpha_j v_j) = \sum_j \alpha_j {\color{darkgreen}Tv_j} \]
}

The matrix is a \emph{representation} for a linear map, and it \emph{depends on the choice of bases}.
Suppose $v \in V$ is an arbitrary vector, while $v_1,\dots,v_n$ is a basis of $V$. The matrix of $v$ wrt this basis is the $n\times 1$ matrix:
%
\[
\mathbf{v} = \begin{pmatrix}
    c_1 \\
    \vdots\\
    c_n
  \end{pmatrix}
\]
%
so that
%
\[
v = c_1 v_1 + \cdots c_n v_n
\]

Once again, we see that the matrix \emph{depends on the choice of basis} for $V$


\begin{itemize}
\item \textbf{addition}:  the matrix of $S+T$ can be obtained by summing the matrices of $S$ and $T${; this only makes sense if the \emph{same bases} are used for $S$, $T$, and $S+T$}
\item \textbf{scalar multiplication}: given $\lambda\in\mathbb{R}$, the matrix for $\lambda T$ is given by $\lambda$ times the matrix of $T$
\end{itemize}

{
In fact, we have just shown that \emph{matrices form a vector space} (Q1: what is the additive identity?) {(Q2: what is the vector space dimension?)}
}

{
We call $\mathbb{R}^{m\times n}$ the vector space of all $m\times n$ matrices with values in $\mathbb{R}$
}

{
\begin{itemize}
\item \textbf{product}: the matrix for $ST$ can be computed by the \emph{matrix product} between $\mathbf{S}$ and $\mathbf{T}$; in fact, the matrix product is defined precisely to make this work

{
Q3: is matrix product commutative?
}

{
Q4: do we need the same bases for $S:U\to V$ and $T:V \to W$?
}

\end{itemize}
}



Consider a linear map $T:V \to W$, a basis $v_1,\dots,v_n \in V$ and a basis $w_1,\dots,w_m\in W$.

%\medskip
%The $k$-th column of $\mathbf{T}$ equals the matrix vector $\mathbf{v}_k$:

%figure

From the definition of matrix product, one can show that it operates on a vector matrix as expected:
\[
\mathbf{Tv} =\mathbf{w} \quad \Leftrightarrow \quad Tv=w
\]
where $\mathbf{Tv}$ is the matrix product of $\mathbf{T}$ and $\mathbf{v}$, while $Tv$ simply denotes the function evaluation $T(v)$


{
\textbf{Remember:} $\mathbf{T}, \mathbf{v}, \mathbf{w}$ must follow a coherent choice of bases in order for the above to make sense. $\mathbf{v}$ can not be expressed in basis $\color{red}(\tilde{v}_1,\dots,\tilde{v}_n)$ if $\mathbf{T}$ only knows how to map basis vectors $\color{blue}({v}_1,\dots,{v}_n)$.
%
\[
T{\color{blue}v_j} = {\color{blue}T_{1,j}} w_1 + \cdots + {\color{blue}T_{m,j}} w_m
\]
%
%\[ Tv =  \sum_j \alpha_j T{\color{red}\tilde{v}_j} \]
}


\[
\underbrace{
\begin{pmatrix}
    T_{1,1} & \cdots & T_{1,n} \\
    \vdots & & \vdots\\
    T_{m,1} & \cdots & T_{m,n}
  \end{pmatrix}}_{\mathbf{T}}
  %
  \underbrace{
  \begin{pmatrix}
    c_1 \\
    \vdots \\
    c_n
  \end{pmatrix}}_{\mathbf{c}} =
  %
   \sum_{j=1}^n c_j \hspace{-0.6cm}
   \underbrace{\begin{pmatrix}
    {\color{red}T_{1,j}}  \\
    \vdots \\
    {\color{red}T_{m,j}}
  \end{pmatrix}}_{\mathrm{Tv_j~wrt~(w_1,\dots,w_m)}}
\]
%
\smallskip

Because recall that, for bases $v_1,\dots,v_n \in V$ and $w_1,\dots,w_m\in W$:
%
\[
Tv_j = {\color{red}T_{1,j}} w_1 + \cdots + {\color{red}T_{m,j}} w_m
\]

{
We see then that vector $c=\sum_j c_j v_j$ is mapped to $Tc = \sum_j c_j Tv_j$.

In other words, matrix product is behaving as expected.
}









\chapter{Geometric deep learning} 
%!TEX root=../../../root.tex

Consider a linear map $T:V \to W$, a basis $v_1,\dots,v_n \in V$ and a basis $w_1,\dots,w_m\in W$.
{
The \emph{matrix} of $T$ in these bases is the $m\times n$ array of values in $\mathbb{R}$
%
\[
\mathbf{T} = \begin{pmatrix}
    T_{1,1} & \cdots & T_{1,n} \\
    \vdots & & \vdots\\
    T_{m,1} & \cdots & T_{m,n}
  \end{pmatrix}
\]
%
whose entries $T_{i,j}$ are defined by
%
\[
{\color{darkgreen}Tv_j} = {\color{red}T_{1,j}} w_1 + \cdots + {\color{red}T_{m,j}} w_m
\]
}%

{
Hence each column of $\mathbf{T}$ contains the {\color{red}linear combination coefficients} for the {\color{darkgreen}image via $T$ of a basis vector from $V$}
}

{
In other words, the matrix encodes {\color{darkgreen}how basis vectors are mapped}, and this is enough to map all other vectors in their span, since:
\[ Tv = T ( \sum_j \alpha_j v_j ) = \sum_j T(\alpha_j v_j) = \sum_j \alpha_j {\color{darkgreen}Tv_j} \]
}

The matrix is a \emph{representation} for a linear map, and it \emph{depends on the choice of bases}.
Suppose $v \in V$ is an arbitrary vector, while $v_1,\dots,v_n$ is a basis of $V$. The matrix of $v$ wrt this basis is the $n\times 1$ matrix:
%
\[
\mathbf{v} = \begin{pmatrix}
    c_1 \\
    \vdots\\
    c_n
  \end{pmatrix}
\]
%
so that
%
\[
v = c_1 v_1 + \cdots c_n v_n
\]

Once again, we see that the matrix \emph{depends on the choice of basis} for $V$


\begin{itemize}
\item \textbf{addition}:  the matrix of $S+T$ can be obtained by summing the matrices of $S$ and $T${; this only makes sense if the \emph{same bases} are used for $S$, $T$, and $S+T$}
\item \textbf{scalar multiplication}: given $\lambda\in\mathbb{R}$, the matrix for $\lambda T$ is given by $\lambda$ times the matrix of $T$
\end{itemize}

{
In fact, we have just shown that \emph{matrices form a vector space} (Q1: what is the additive identity?) {(Q2: what is the vector space dimension?)}
}

{
We call $\mathbb{R}^{m\times n}$ the vector space of all $m\times n$ matrices with values in $\mathbb{R}$
}

{
\begin{itemize}
\item \textbf{product}: the matrix for $ST$ can be computed by the \emph{matrix product} between $\mathbf{S}$ and $\mathbf{T}$; in fact, the matrix product is defined precisely to make this work

{
Q3: is matrix product commutative?
}

{
Q4: do we need the same bases for $S:U\to V$ and $T:V \to W$?
}

\end{itemize}
}



Consider a linear map $T:V \to W$, a basis $v_1,\dots,v_n \in V$ and a basis $w_1,\dots,w_m\in W$.

%\medskip
%The $k$-th column of $\mathbf{T}$ equals the matrix vector $\mathbf{v}_k$:

%figure

From the definition of matrix product, one can show that it operates on a vector matrix as expected:
\[
\mathbf{Tv} =\mathbf{w} \quad \Leftrightarrow \quad Tv=w
\]
where $\mathbf{Tv}$ is the matrix product of $\mathbf{T}$ and $\mathbf{v}$, while $Tv$ simply denotes the function evaluation $T(v)$


{
\textbf{Remember:} $\mathbf{T}, \mathbf{v}, \mathbf{w}$ must follow a coherent choice of bases in order for the above to make sense. $\mathbf{v}$ can not be expressed in basis $\color{red}(\tilde{v}_1,\dots,\tilde{v}_n)$ if $\mathbf{T}$ only knows how to map basis vectors $\color{blue}({v}_1,\dots,{v}_n)$.
%
\[
T{\color{blue}v_j} = {\color{blue}T_{1,j}} w_1 + \cdots + {\color{blue}T_{m,j}} w_m
\]
%
%\[ Tv =  \sum_j \alpha_j T{\color{red}\tilde{v}_j} \]
}


\[
\underbrace{
\begin{pmatrix}
    T_{1,1} & \cdots & T_{1,n} \\
    \vdots & & \vdots\\
    T_{m,1} & \cdots & T_{m,n}
  \end{pmatrix}}_{\mathbf{T}}
  %
  \underbrace{
  \begin{pmatrix}
    c_1 \\
    \vdots \\
    c_n
  \end{pmatrix}}_{\mathbf{c}} =
  %
   \sum_{j=1}^n c_j \hspace{-0.6cm}
   \underbrace{\begin{pmatrix}
    {\color{red}T_{1,j}}  \\
    \vdots \\
    {\color{red}T_{m,j}}
  \end{pmatrix}}_{\mathrm{Tv_j~wrt~(w_1,\dots,w_m)}}
\]
%
\smallskip

Because recall that, for bases $v_1,\dots,v_n \in V$ and $w_1,\dots,w_m\in W$:
%
\[
Tv_j = {\color{red}T_{1,j}} w_1 + \cdots + {\color{red}T_{m,j}} w_m
\]

{
We see then that vector $c=\sum_j c_j v_j$ is mapped to $Tc = \sum_j c_j Tv_j$.

In other words, matrix product is behaving as expected.
}









\chapter{Adversarial training} 
%!TEX root=../../../root.tex

Consider a linear map $T:V \to W$, a basis $v_1,\dots,v_n \in V$ and a basis $w_1,\dots,w_m\in W$.
{
The \emph{matrix} of $T$ in these bases is the $m\times n$ array of values in $\mathbb{R}$
%
\[
\mathbf{T} = \begin{pmatrix}
    T_{1,1} & \cdots & T_{1,n} \\
    \vdots & & \vdots\\
    T_{m,1} & \cdots & T_{m,n}
  \end{pmatrix}
\]
%
whose entries $T_{i,j}$ are defined by
%
\[
{\color{darkgreen}Tv_j} = {\color{red}T_{1,j}} w_1 + \cdots + {\color{red}T_{m,j}} w_m
\]
}%

{
Hence each column of $\mathbf{T}$ contains the {\color{red}linear combination coefficients} for the {\color{darkgreen}image via $T$ of a basis vector from $V$}
}

{
In other words, the matrix encodes {\color{darkgreen}how basis vectors are mapped}, and this is enough to map all other vectors in their span, since:
\[ Tv = T ( \sum_j \alpha_j v_j ) = \sum_j T(\alpha_j v_j) = \sum_j \alpha_j {\color{darkgreen}Tv_j} \]
}

The matrix is a \emph{representation} for a linear map, and it \emph{depends on the choice of bases}.
Suppose $v \in V$ is an arbitrary vector, while $v_1,\dots,v_n$ is a basis of $V$. The matrix of $v$ wrt this basis is the $n\times 1$ matrix:
%
\[
\mathbf{v} = \begin{pmatrix}
    c_1 \\
    \vdots\\
    c_n
  \end{pmatrix}
\]
%
so that
%
\[
v = c_1 v_1 + \cdots c_n v_n
\]

Once again, we see that the matrix \emph{depends on the choice of basis} for $V$


\begin{itemize}
\item \textbf{addition}:  the matrix of $S+T$ can be obtained by summing the matrices of $S$ and $T${; this only makes sense if the \emph{same bases} are used for $S$, $T$, and $S+T$}
\item \textbf{scalar multiplication}: given $\lambda\in\mathbb{R}$, the matrix for $\lambda T$ is given by $\lambda$ times the matrix of $T$
\end{itemize}

{
In fact, we have just shown that \emph{matrices form a vector space} (Q1: what is the additive identity?) {(Q2: what is the vector space dimension?)}
}

{
We call $\mathbb{R}^{m\times n}$ the vector space of all $m\times n$ matrices with values in $\mathbb{R}$
}

{
\begin{itemize}
\item \textbf{product}: the matrix for $ST$ can be computed by the \emph{matrix product} between $\mathbf{S}$ and $\mathbf{T}$; in fact, the matrix product is defined precisely to make this work

{
Q3: is matrix product commutative?
}

{
Q4: do we need the same bases for $S:U\to V$ and $T:V \to W$?
}

\end{itemize}
}



Consider a linear map $T:V \to W$, a basis $v_1,\dots,v_n \in V$ and a basis $w_1,\dots,w_m\in W$.

%\medskip
%The $k$-th column of $\mathbf{T}$ equals the matrix vector $\mathbf{v}_k$:

%figure

From the definition of matrix product, one can show that it operates on a vector matrix as expected:
\[
\mathbf{Tv} =\mathbf{w} \quad \Leftrightarrow \quad Tv=w
\]
where $\mathbf{Tv}$ is the matrix product of $\mathbf{T}$ and $\mathbf{v}$, while $Tv$ simply denotes the function evaluation $T(v)$


{
\textbf{Remember:} $\mathbf{T}, \mathbf{v}, \mathbf{w}$ must follow a coherent choice of bases in order for the above to make sense. $\mathbf{v}$ can not be expressed in basis $\color{red}(\tilde{v}_1,\dots,\tilde{v}_n)$ if $\mathbf{T}$ only knows how to map basis vectors $\color{blue}({v}_1,\dots,{v}_n)$.
%
\[
T{\color{blue}v_j} = {\color{blue}T_{1,j}} w_1 + \cdots + {\color{blue}T_{m,j}} w_m
\]
%
%\[ Tv =  \sum_j \alpha_j T{\color{red}\tilde{v}_j} \]
}


\[
\underbrace{
\begin{pmatrix}
    T_{1,1} & \cdots & T_{1,n} \\
    \vdots & & \vdots\\
    T_{m,1} & \cdots & T_{m,n}
  \end{pmatrix}}_{\mathbf{T}}
  %
  \underbrace{
  \begin{pmatrix}
    c_1 \\
    \vdots \\
    c_n
  \end{pmatrix}}_{\mathbf{c}} =
  %
   \sum_{j=1}^n c_j \hspace{-0.6cm}
   \underbrace{\begin{pmatrix}
    {\color{red}T_{1,j}}  \\
    \vdots \\
    {\color{red}T_{m,j}}
  \end{pmatrix}}_{\mathrm{Tv_j~wrt~(w_1,\dots,w_m)}}
\]
%
\smallskip

Because recall that, for bases $v_1,\dots,v_n \in V$ and $w_1,\dots,w_m\in W$:
%
\[
Tv_j = {\color{red}T_{1,j}} w_1 + \cdots + {\color{red}T_{m,j}} w_m
\]

{
We see then that vector $c=\sum_j c_j v_j$ is mapped to $Tc = \sum_j c_j Tv_j$.

In other words, matrix product is behaving as expected.
}









% \chapter{Transformers}
% %!TEX root=../../../root.tex

Consider a linear map $T:V \to W$, a basis $v_1,\dots,v_n \in V$ and a basis $w_1,\dots,w_m\in W$.
{
The \emph{matrix} of $T$ in these bases is the $m\times n$ array of values in $\mathbb{R}$
%
\[
\mathbf{T} = \begin{pmatrix}
    T_{1,1} & \cdots & T_{1,n} \\
    \vdots & & \vdots\\
    T_{m,1} & \cdots & T_{m,n}
  \end{pmatrix}
\]
%
whose entries $T_{i,j}$ are defined by
%
\[
{\color{darkgreen}Tv_j} = {\color{red}T_{1,j}} w_1 + \cdots + {\color{red}T_{m,j}} w_m
\]
}%

{
Hence each column of $\mathbf{T}$ contains the {\color{red}linear combination coefficients} for the {\color{darkgreen}image via $T$ of a basis vector from $V$}
}

{
In other words, the matrix encodes {\color{darkgreen}how basis vectors are mapped}, and this is enough to map all other vectors in their span, since:
\[ Tv = T ( \sum_j \alpha_j v_j ) = \sum_j T(\alpha_j v_j) = \sum_j \alpha_j {\color{darkgreen}Tv_j} \]
}

The matrix is a \emph{representation} for a linear map, and it \emph{depends on the choice of bases}.
Suppose $v \in V$ is an arbitrary vector, while $v_1,\dots,v_n$ is a basis of $V$. The matrix of $v$ wrt this basis is the $n\times 1$ matrix:
%
\[
\mathbf{v} = \begin{pmatrix}
    c_1 \\
    \vdots\\
    c_n
  \end{pmatrix}
\]
%
so that
%
\[
v = c_1 v_1 + \cdots c_n v_n
\]

Once again, we see that the matrix \emph{depends on the choice of basis} for $V$


\begin{itemize}
\item \textbf{addition}:  the matrix of $S+T$ can be obtained by summing the matrices of $S$ and $T${; this only makes sense if the \emph{same bases} are used for $S$, $T$, and $S+T$}
\item \textbf{scalar multiplication}: given $\lambda\in\mathbb{R}$, the matrix for $\lambda T$ is given by $\lambda$ times the matrix of $T$
\end{itemize}

{
In fact, we have just shown that \emph{matrices form a vector space} (Q1: what is the additive identity?) {(Q2: what is the vector space dimension?)}
}

{
We call $\mathbb{R}^{m\times n}$ the vector space of all $m\times n$ matrices with values in $\mathbb{R}$
}

{
\begin{itemize}
\item \textbf{product}: the matrix for $ST$ can be computed by the \emph{matrix product} between $\mathbf{S}$ and $\mathbf{T}$; in fact, the matrix product is defined precisely to make this work

{
Q3: is matrix product commutative?
}

{
Q4: do we need the same bases for $S:U\to V$ and $T:V \to W$?
}

\end{itemize}
}



Consider a linear map $T:V \to W$, a basis $v_1,\dots,v_n \in V$ and a basis $w_1,\dots,w_m\in W$.

%\medskip
%The $k$-th column of $\mathbf{T}$ equals the matrix vector $\mathbf{v}_k$:

%figure

From the definition of matrix product, one can show that it operates on a vector matrix as expected:
\[
\mathbf{Tv} =\mathbf{w} \quad \Leftrightarrow \quad Tv=w
\]
where $\mathbf{Tv}$ is the matrix product of $\mathbf{T}$ and $\mathbf{v}$, while $Tv$ simply denotes the function evaluation $T(v)$


{
\textbf{Remember:} $\mathbf{T}, \mathbf{v}, \mathbf{w}$ must follow a coherent choice of bases in order for the above to make sense. $\mathbf{v}$ can not be expressed in basis $\color{red}(\tilde{v}_1,\dots,\tilde{v}_n)$ if $\mathbf{T}$ only knows how to map basis vectors $\color{blue}({v}_1,\dots,{v}_n)$.
%
\[
T{\color{blue}v_j} = {\color{blue}T_{1,j}} w_1 + \cdots + {\color{blue}T_{m,j}} w_m
\]
%
%\[ Tv =  \sum_j \alpha_j T{\color{red}\tilde{v}_j} \]
}


\[
\underbrace{
\begin{pmatrix}
    T_{1,1} & \cdots & T_{1,n} \\
    \vdots & & \vdots\\
    T_{m,1} & \cdots & T_{m,n}
  \end{pmatrix}}_{\mathbf{T}}
  %
  \underbrace{
  \begin{pmatrix}
    c_1 \\
    \vdots \\
    c_n
  \end{pmatrix}}_{\mathbf{c}} =
  %
   \sum_{j=1}^n c_j \hspace{-0.6cm}
   \underbrace{\begin{pmatrix}
    {\color{red}T_{1,j}}  \\
    \vdots \\
    {\color{red}T_{m,j}}
  \end{pmatrix}}_{\mathrm{Tv_j~wrt~(w_1,\dots,w_m)}}
\]
%
\smallskip

Because recall that, for bases $v_1,\dots,v_n \in V$ and $w_1,\dots,w_m\in W$:
%
\[
Tv_j = {\color{red}T_{1,j}} w_1 + \cdots + {\color{red}T_{m,j}} w_m
\]

{
We see then that vector $c=\sum_j c_j v_j$ is mapped to $Tc = \sum_j c_j Tv_j$.

In other words, matrix product is behaving as expected.
}









\appendix

\chapter{Linear Algebra} 
%!TEX root=../../../root.tex

Consider a linear map $T:V \to W$, a basis $v_1,\dots,v_n \in V$ and a basis $w_1,\dots,w_m\in W$.
{
The \emph{matrix} of $T$ in these bases is the $m\times n$ array of values in $\mathbb{R}$
%
\[
\mathbf{T} = \begin{pmatrix}
    T_{1,1} & \cdots & T_{1,n} \\
    \vdots & & \vdots\\
    T_{m,1} & \cdots & T_{m,n}
  \end{pmatrix}
\]
%
whose entries $T_{i,j}$ are defined by
%
\[
{\color{darkgreen}Tv_j} = {\color{red}T_{1,j}} w_1 + \cdots + {\color{red}T_{m,j}} w_m
\]
}%

{
Hence each column of $\mathbf{T}$ contains the {\color{red}linear combination coefficients} for the {\color{darkgreen}image via $T$ of a basis vector from $V$}
}

{
In other words, the matrix encodes {\color{darkgreen}how basis vectors are mapped}, and this is enough to map all other vectors in their span, since:
\[ Tv = T ( \sum_j \alpha_j v_j ) = \sum_j T(\alpha_j v_j) = \sum_j \alpha_j {\color{darkgreen}Tv_j} \]
}

The matrix is a \emph{representation} for a linear map, and it \emph{depends on the choice of bases}.
Suppose $v \in V$ is an arbitrary vector, while $v_1,\dots,v_n$ is a basis of $V$. The matrix of $v$ wrt this basis is the $n\times 1$ matrix:
%
\[
\mathbf{v} = \begin{pmatrix}
    c_1 \\
    \vdots\\
    c_n
  \end{pmatrix}
\]
%
so that
%
\[
v = c_1 v_1 + \cdots c_n v_n
\]

Once again, we see that the matrix \emph{depends on the choice of basis} for $V$


\begin{itemize}
\item \textbf{addition}:  the matrix of $S+T$ can be obtained by summing the matrices of $S$ and $T${; this only makes sense if the \emph{same bases} are used for $S$, $T$, and $S+T$}
\item \textbf{scalar multiplication}: given $\lambda\in\mathbb{R}$, the matrix for $\lambda T$ is given by $\lambda$ times the matrix of $T$
\end{itemize}

{
In fact, we have just shown that \emph{matrices form a vector space} (Q1: what is the additive identity?) {(Q2: what is the vector space dimension?)}
}

{
We call $\mathbb{R}^{m\times n}$ the vector space of all $m\times n$ matrices with values in $\mathbb{R}$
}

{
\begin{itemize}
\item \textbf{product}: the matrix for $ST$ can be computed by the \emph{matrix product} between $\mathbf{S}$ and $\mathbf{T}$; in fact, the matrix product is defined precisely to make this work

{
Q3: is matrix product commutative?
}

{
Q4: do we need the same bases for $S:U\to V$ and $T:V \to W$?
}

\end{itemize}
}



Consider a linear map $T:V \to W$, a basis $v_1,\dots,v_n \in V$ and a basis $w_1,\dots,w_m\in W$.

%\medskip
%The $k$-th column of $\mathbf{T}$ equals the matrix vector $\mathbf{v}_k$:

%figure

From the definition of matrix product, one can show that it operates on a vector matrix as expected:
\[
\mathbf{Tv} =\mathbf{w} \quad \Leftrightarrow \quad Tv=w
\]
where $\mathbf{Tv}$ is the matrix product of $\mathbf{T}$ and $\mathbf{v}$, while $Tv$ simply denotes the function evaluation $T(v)$


{
\textbf{Remember:} $\mathbf{T}, \mathbf{v}, \mathbf{w}$ must follow a coherent choice of bases in order for the above to make sense. $\mathbf{v}$ can not be expressed in basis $\color{red}(\tilde{v}_1,\dots,\tilde{v}_n)$ if $\mathbf{T}$ only knows how to map basis vectors $\color{blue}({v}_1,\dots,{v}_n)$.
%
\[
T{\color{blue}v_j} = {\color{blue}T_{1,j}} w_1 + \cdots + {\color{blue}T_{m,j}} w_m
\]
%
%\[ Tv =  \sum_j \alpha_j T{\color{red}\tilde{v}_j} \]
}


\[
\underbrace{
\begin{pmatrix}
    T_{1,1} & \cdots & T_{1,n} \\
    \vdots & & \vdots\\
    T_{m,1} & \cdots & T_{m,n}
  \end{pmatrix}}_{\mathbf{T}}
  %
  \underbrace{
  \begin{pmatrix}
    c_1 \\
    \vdots \\
    c_n
  \end{pmatrix}}_{\mathbf{c}} =
  %
   \sum_{j=1}^n c_j \hspace{-0.6cm}
   \underbrace{\begin{pmatrix}
    {\color{red}T_{1,j}}  \\
    \vdots \\
    {\color{red}T_{m,j}}
  \end{pmatrix}}_{\mathrm{Tv_j~wrt~(w_1,\dots,w_m)}}
\]
%
\smallskip

Because recall that, for bases $v_1,\dots,v_n \in V$ and $w_1,\dots,w_m\in W$:
%
\[
Tv_j = {\color{red}T_{1,j}} w_1 + \cdots + {\color{red}T_{m,j}} w_m
\]

{
We see then that vector $c=\sum_j c_j v_j$ is mapped to $Tc = \sum_j c_j Tv_j$.

In other words, matrix product is behaving as expected.
}








\chapter{Information Theory} \label{sec:appendix-A}
%!TEX root=../../../root.tex

Consider a linear map $T:V \to W$, a basis $v_1,\dots,v_n \in V$ and a basis $w_1,\dots,w_m\in W$.
{
The \emph{matrix} of $T$ in these bases is the $m\times n$ array of values in $\mathbb{R}$
%
\[
\mathbf{T} = \begin{pmatrix}
    T_{1,1} & \cdots & T_{1,n} \\
    \vdots & & \vdots\\
    T_{m,1} & \cdots & T_{m,n}
  \end{pmatrix}
\]
%
whose entries $T_{i,j}$ are defined by
%
\[
{\color{darkgreen}Tv_j} = {\color{red}T_{1,j}} w_1 + \cdots + {\color{red}T_{m,j}} w_m
\]
}%

{
Hence each column of $\mathbf{T}$ contains the {\color{red}linear combination coefficients} for the {\color{darkgreen}image via $T$ of a basis vector from $V$}
}

{
In other words, the matrix encodes {\color{darkgreen}how basis vectors are mapped}, and this is enough to map all other vectors in their span, since:
\[ Tv = T ( \sum_j \alpha_j v_j ) = \sum_j T(\alpha_j v_j) = \sum_j \alpha_j {\color{darkgreen}Tv_j} \]
}

The matrix is a \emph{representation} for a linear map, and it \emph{depends on the choice of bases}.
Suppose $v \in V$ is an arbitrary vector, while $v_1,\dots,v_n$ is a basis of $V$. The matrix of $v$ wrt this basis is the $n\times 1$ matrix:
%
\[
\mathbf{v} = \begin{pmatrix}
    c_1 \\
    \vdots\\
    c_n
  \end{pmatrix}
\]
%
so that
%
\[
v = c_1 v_1 + \cdots c_n v_n
\]

Once again, we see that the matrix \emph{depends on the choice of basis} for $V$


\begin{itemize}
\item \textbf{addition}:  the matrix of $S+T$ can be obtained by summing the matrices of $S$ and $T${; this only makes sense if the \emph{same bases} are used for $S$, $T$, and $S+T$}
\item \textbf{scalar multiplication}: given $\lambda\in\mathbb{R}$, the matrix for $\lambda T$ is given by $\lambda$ times the matrix of $T$
\end{itemize}

{
In fact, we have just shown that \emph{matrices form a vector space} (Q1: what is the additive identity?) {(Q2: what is the vector space dimension?)}
}

{
We call $\mathbb{R}^{m\times n}$ the vector space of all $m\times n$ matrices with values in $\mathbb{R}$
}

{
\begin{itemize}
\item \textbf{product}: the matrix for $ST$ can be computed by the \emph{matrix product} between $\mathbf{S}$ and $\mathbf{T}$; in fact, the matrix product is defined precisely to make this work

{
Q3: is matrix product commutative?
}

{
Q4: do we need the same bases for $S:U\to V$ and $T:V \to W$?
}

\end{itemize}
}



Consider a linear map $T:V \to W$, a basis $v_1,\dots,v_n \in V$ and a basis $w_1,\dots,w_m\in W$.

%\medskip
%The $k$-th column of $\mathbf{T}$ equals the matrix vector $\mathbf{v}_k$:

%figure

From the definition of matrix product, one can show that it operates on a vector matrix as expected:
\[
\mathbf{Tv} =\mathbf{w} \quad \Leftrightarrow \quad Tv=w
\]
where $\mathbf{Tv}$ is the matrix product of $\mathbf{T}$ and $\mathbf{v}$, while $Tv$ simply denotes the function evaluation $T(v)$


{
\textbf{Remember:} $\mathbf{T}, \mathbf{v}, \mathbf{w}$ must follow a coherent choice of bases in order for the above to make sense. $\mathbf{v}$ can not be expressed in basis $\color{red}(\tilde{v}_1,\dots,\tilde{v}_n)$ if $\mathbf{T}$ only knows how to map basis vectors $\color{blue}({v}_1,\dots,{v}_n)$.
%
\[
T{\color{blue}v_j} = {\color{blue}T_{1,j}} w_1 + \cdots + {\color{blue}T_{m,j}} w_m
\]
%
%\[ Tv =  \sum_j \alpha_j T{\color{red}\tilde{v}_j} \]
}


\[
\underbrace{
\begin{pmatrix}
    T_{1,1} & \cdots & T_{1,n} \\
    \vdots & & \vdots\\
    T_{m,1} & \cdots & T_{m,n}
  \end{pmatrix}}_{\mathbf{T}}
  %
  \underbrace{
  \begin{pmatrix}
    c_1 \\
    \vdots \\
    c_n
  \end{pmatrix}}_{\mathbf{c}} =
  %
   \sum_{j=1}^n c_j \hspace{-0.6cm}
   \underbrace{\begin{pmatrix}
    {\color{red}T_{1,j}}  \\
    \vdots \\
    {\color{red}T_{m,j}}
  \end{pmatrix}}_{\mathrm{Tv_j~wrt~(w_1,\dots,w_m)}}
\]
%
\smallskip

Because recall that, for bases $v_1,\dots,v_n \in V$ and $w_1,\dots,w_m\in W$:
%
\[
Tv_j = {\color{red}T_{1,j}} w_1 + \cdots + {\color{red}T_{m,j}} w_m
\]

{
We see then that vector $c=\sum_j c_j v_j$ is mapped to $Tc = \sum_j c_j Tv_j$.

In other words, matrix product is behaving as expected.
}








\chapter{Fourier Analysis} \label{sec:appendix:fourier}
%!TEX root=../../../root.tex

Consider a linear map $T:V \to W$, a basis $v_1,\dots,v_n \in V$ and a basis $w_1,\dots,w_m\in W$.
{
The \emph{matrix} of $T$ in these bases is the $m\times n$ array of values in $\mathbb{R}$
%
\[
\mathbf{T} = \begin{pmatrix}
    T_{1,1} & \cdots & T_{1,n} \\
    \vdots & & \vdots\\
    T_{m,1} & \cdots & T_{m,n}
  \end{pmatrix}
\]
%
whose entries $T_{i,j}$ are defined by
%
\[
{\color{darkgreen}Tv_j} = {\color{red}T_{1,j}} w_1 + \cdots + {\color{red}T_{m,j}} w_m
\]
}%

{
Hence each column of $\mathbf{T}$ contains the {\color{red}linear combination coefficients} for the {\color{darkgreen}image via $T$ of a basis vector from $V$}
}

{
In other words, the matrix encodes {\color{darkgreen}how basis vectors are mapped}, and this is enough to map all other vectors in their span, since:
\[ Tv = T ( \sum_j \alpha_j v_j ) = \sum_j T(\alpha_j v_j) = \sum_j \alpha_j {\color{darkgreen}Tv_j} \]
}

The matrix is a \emph{representation} for a linear map, and it \emph{depends on the choice of bases}.
Suppose $v \in V$ is an arbitrary vector, while $v_1,\dots,v_n$ is a basis of $V$. The matrix of $v$ wrt this basis is the $n\times 1$ matrix:
%
\[
\mathbf{v} = \begin{pmatrix}
    c_1 \\
    \vdots\\
    c_n
  \end{pmatrix}
\]
%
so that
%
\[
v = c_1 v_1 + \cdots c_n v_n
\]

Once again, we see that the matrix \emph{depends on the choice of basis} for $V$


\begin{itemize}
\item \textbf{addition}:  the matrix of $S+T$ can be obtained by summing the matrices of $S$ and $T${; this only makes sense if the \emph{same bases} are used for $S$, $T$, and $S+T$}
\item \textbf{scalar multiplication}: given $\lambda\in\mathbb{R}$, the matrix for $\lambda T$ is given by $\lambda$ times the matrix of $T$
\end{itemize}

{
In fact, we have just shown that \emph{matrices form a vector space} (Q1: what is the additive identity?) {(Q2: what is the vector space dimension?)}
}

{
We call $\mathbb{R}^{m\times n}$ the vector space of all $m\times n$ matrices with values in $\mathbb{R}$
}

{
\begin{itemize}
\item \textbf{product}: the matrix for $ST$ can be computed by the \emph{matrix product} between $\mathbf{S}$ and $\mathbf{T}$; in fact, the matrix product is defined precisely to make this work

{
Q3: is matrix product commutative?
}

{
Q4: do we need the same bases for $S:U\to V$ and $T:V \to W$?
}

\end{itemize}
}



Consider a linear map $T:V \to W$, a basis $v_1,\dots,v_n \in V$ and a basis $w_1,\dots,w_m\in W$.

%\medskip
%The $k$-th column of $\mathbf{T}$ equals the matrix vector $\mathbf{v}_k$:

%figure

From the definition of matrix product, one can show that it operates on a vector matrix as expected:
\[
\mathbf{Tv} =\mathbf{w} \quad \Leftrightarrow \quad Tv=w
\]
where $\mathbf{Tv}$ is the matrix product of $\mathbf{T}$ and $\mathbf{v}$, while $Tv$ simply denotes the function evaluation $T(v)$


{
\textbf{Remember:} $\mathbf{T}, \mathbf{v}, \mathbf{w}$ must follow a coherent choice of bases in order for the above to make sense. $\mathbf{v}$ can not be expressed in basis $\color{red}(\tilde{v}_1,\dots,\tilde{v}_n)$ if $\mathbf{T}$ only knows how to map basis vectors $\color{blue}({v}_1,\dots,{v}_n)$.
%
\[
T{\color{blue}v_j} = {\color{blue}T_{1,j}} w_1 + \cdots + {\color{blue}T_{m,j}} w_m
\]
%
%\[ Tv =  \sum_j \alpha_j T{\color{red}\tilde{v}_j} \]
}


\[
\underbrace{
\begin{pmatrix}
    T_{1,1} & \cdots & T_{1,n} \\
    \vdots & & \vdots\\
    T_{m,1} & \cdots & T_{m,n}
  \end{pmatrix}}_{\mathbf{T}}
  %
  \underbrace{
  \begin{pmatrix}
    c_1 \\
    \vdots \\
    c_n
  \end{pmatrix}}_{\mathbf{c}} =
  %
   \sum_{j=1}^n c_j \hspace{-0.6cm}
   \underbrace{\begin{pmatrix}
    {\color{red}T_{1,j}}  \\
    \vdots \\
    {\color{red}T_{m,j}}
  \end{pmatrix}}_{\mathrm{Tv_j~wrt~(w_1,\dots,w_m)}}
\]
%
\smallskip

Because recall that, for bases $v_1,\dots,v_n \in V$ and $w_1,\dots,w_m\in W$:
%
\[
Tv_j = {\color{red}T_{1,j}} w_1 + \cdots + {\color{red}T_{m,j}} w_m
\]

{
We see then that vector $c=\sum_j c_j v_j$ is mapped to $Tc = \sum_j c_j Tv_j$.

In other words, matrix product is behaving as expected.
}








\chapter{Spectral Graph Theory} 
%!TEX root=../../../root.tex

Consider a linear map $T:V \to W$, a basis $v_1,\dots,v_n \in V$ and a basis $w_1,\dots,w_m\in W$.
{
The \emph{matrix} of $T$ in these bases is the $m\times n$ array of values in $\mathbb{R}$
%
\[
\mathbf{T} = \begin{pmatrix}
    T_{1,1} & \cdots & T_{1,n} \\
    \vdots & & \vdots\\
    T_{m,1} & \cdots & T_{m,n}
  \end{pmatrix}
\]
%
whose entries $T_{i,j}$ are defined by
%
\[
{\color{darkgreen}Tv_j} = {\color{red}T_{1,j}} w_1 + \cdots + {\color{red}T_{m,j}} w_m
\]
}%

{
Hence each column of $\mathbf{T}$ contains the {\color{red}linear combination coefficients} for the {\color{darkgreen}image via $T$ of a basis vector from $V$}
}

{
In other words, the matrix encodes {\color{darkgreen}how basis vectors are mapped}, and this is enough to map all other vectors in their span, since:
\[ Tv = T ( \sum_j \alpha_j v_j ) = \sum_j T(\alpha_j v_j) = \sum_j \alpha_j {\color{darkgreen}Tv_j} \]
}

The matrix is a \emph{representation} for a linear map, and it \emph{depends on the choice of bases}.
Suppose $v \in V$ is an arbitrary vector, while $v_1,\dots,v_n$ is a basis of $V$. The matrix of $v$ wrt this basis is the $n\times 1$ matrix:
%
\[
\mathbf{v} = \begin{pmatrix}
    c_1 \\
    \vdots\\
    c_n
  \end{pmatrix}
\]
%
so that
%
\[
v = c_1 v_1 + \cdots c_n v_n
\]

Once again, we see that the matrix \emph{depends on the choice of basis} for $V$


\begin{itemize}
\item \textbf{addition}:  the matrix of $S+T$ can be obtained by summing the matrices of $S$ and $T${; this only makes sense if the \emph{same bases} are used for $S$, $T$, and $S+T$}
\item \textbf{scalar multiplication}: given $\lambda\in\mathbb{R}$, the matrix for $\lambda T$ is given by $\lambda$ times the matrix of $T$
\end{itemize}

{
In fact, we have just shown that \emph{matrices form a vector space} (Q1: what is the additive identity?) {(Q2: what is the vector space dimension?)}
}

{
We call $\mathbb{R}^{m\times n}$ the vector space of all $m\times n$ matrices with values in $\mathbb{R}$
}

{
\begin{itemize}
\item \textbf{product}: the matrix for $ST$ can be computed by the \emph{matrix product} between $\mathbf{S}$ and $\mathbf{T}$; in fact, the matrix product is defined precisely to make this work

{
Q3: is matrix product commutative?
}

{
Q4: do we need the same bases for $S:U\to V$ and $T:V \to W$?
}

\end{itemize}
}



Consider a linear map $T:V \to W$, a basis $v_1,\dots,v_n \in V$ and a basis $w_1,\dots,w_m\in W$.

%\medskip
%The $k$-th column of $\mathbf{T}$ equals the matrix vector $\mathbf{v}_k$:

%figure

From the definition of matrix product, one can show that it operates on a vector matrix as expected:
\[
\mathbf{Tv} =\mathbf{w} \quad \Leftrightarrow \quad Tv=w
\]
where $\mathbf{Tv}$ is the matrix product of $\mathbf{T}$ and $\mathbf{v}$, while $Tv$ simply denotes the function evaluation $T(v)$


{
\textbf{Remember:} $\mathbf{T}, \mathbf{v}, \mathbf{w}$ must follow a coherent choice of bases in order for the above to make sense. $\mathbf{v}$ can not be expressed in basis $\color{red}(\tilde{v}_1,\dots,\tilde{v}_n)$ if $\mathbf{T}$ only knows how to map basis vectors $\color{blue}({v}_1,\dots,{v}_n)$.
%
\[
T{\color{blue}v_j} = {\color{blue}T_{1,j}} w_1 + \cdots + {\color{blue}T_{m,j}} w_m
\]
%
%\[ Tv =  \sum_j \alpha_j T{\color{red}\tilde{v}_j} \]
}


\[
\underbrace{
\begin{pmatrix}
    T_{1,1} & \cdots & T_{1,n} \\
    \vdots & & \vdots\\
    T_{m,1} & \cdots & T_{m,n}
  \end{pmatrix}}_{\mathbf{T}}
  %
  \underbrace{
  \begin{pmatrix}
    c_1 \\
    \vdots \\
    c_n
  \end{pmatrix}}_{\mathbf{c}} =
  %
   \sum_{j=1}^n c_j \hspace{-0.6cm}
   \underbrace{\begin{pmatrix}
    {\color{red}T_{1,j}}  \\
    \vdots \\
    {\color{red}T_{m,j}}
  \end{pmatrix}}_{\mathrm{Tv_j~wrt~(w_1,\dots,w_m)}}
\]
%
\smallskip

Because recall that, for bases $v_1,\dots,v_n \in V$ and $w_1,\dots,w_m\in W$:
%
\[
Tv_j = {\color{red}T_{1,j}} w_1 + \cdots + {\color{red}T_{m,j}} w_m
\]

{
We see then that vector $c=\sum_j c_j v_j$ is mapped to $Tc = \sum_j c_j Tv_j$.

In other words, matrix product is behaving as expected.
}








\end{document}