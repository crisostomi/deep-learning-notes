%!TEX root=../../../root.tex

Translational invariance is one of the most common and therefore most desirable invariances. In particular, is desirable \emph{across multiple scales}, leading to \emph{compositionality}. In a hierarchical way, we expect the data (we will concentrate on images in this chapter) to be able to be decomposed in \emph{local features} at each scale, invariant of their location in the image:
\begin{equation}
    z({\cal T}_v p) = z(p)  ~~~\forall p,{\cal T}_v
\end{equation}
where $p$ are image patches of variable size.

\begin{figure}[H]
    \centering
    \begin{overpic}
        [trim=0cm 0cm 0cm 0cm,clip,width=0.9\linewidth]{08/hierar}
        \put(11,-2){\footnotesize scale $1$}
        \put(76,-2){\footnotesize scale $n$}
    \end{overpic}
    \caption{A human face is still a human face even when translated, and so is a nose or an eye, and so are the edges that make up each of those, provided that when arranged together, the local features at a lower scale \textit{compose} a correct local feature at a higher scale.}
\end{figure}