%!TEX root=../root.tex

In the previous chapter we have presented the Multi-Layer Perceptron, what most people refer to when speaking about a Neural Network in general. These are \emph{deep} networks, since they are made up of several layers, and are \emph{feed-forward} networks, since the data progresses through the network from the input layer to the output layer in a straight-forward way, undergoing transformations at each layer in the process.

\begin{figure}[H]
    \centering
    \begin{overpic}
	[trim=0cm 0cm 0cm 0cm,clip,width=0.95\linewidth]{08/dnn1_.pdf}
		\put(1.25,30.25){\footnotesize $f_1^\mathrm{in}$}
		\put(1.25,22.5){\footnotesize $f_2^\mathrm{in}$}
		\put(1.25,5.5){\footnotesize $f_n^\mathrm{in}$}
		\put(26.5,33){\footnotesize $g_1^{(1)}$}
		\put(26.5,25.25){\footnotesize $g_2^{(1)}$}
		\put(26.5,8.25){\footnotesize $g_{m_1}^{(1)}$}
		\put(47.75,33){\footnotesize $g_1^{(2)}$}
		\put(47.75,25.25){\footnotesize $g_2^{(2)}$}
		\put(47.75,8.25){\footnotesize $g_{m_2}^{(2)}$}
		\put(69.25,33){\footnotesize $g_1^{(3)}$}
		\put(69.25,25.25){\footnotesize $g_2^{(3)}$}
		\put(69.25,8.25){\footnotesize $g_{m_3}^{(3)}$}
		\put(76.25,33){\footnotesize $g_1^{(L-1)}$}
		\put(76.25,25.25){\footnotesize $g_2^{(L-1)}$}
		\put(76.25,8.25){\footnotesize $g_{m_{L-1}}^{(L-1)}$}
		\put(93.5,30.25){\footnotesize $g_1^\mathrm{out}$}
		\put(93.5,22.5){\footnotesize $g_2^\mathrm{out}$}
		\put(93.5,5.5){\footnotesize $g_m^\mathrm{out}$}	
		\put(15,3){\footnotesize $\mathbf{W}^{(1)}$}			
		\put(36,3){\footnotesize $\mathbf{W}^{(2)}$}	
		\put(57,3){\footnotesize $\mathbf{W}^{(3)}$}	
		\put(82,3){\footnotesize $\mathbf{W}^{(L)}$}					
	\end{overpic}
    \caption{Deep feed-forward neural network consisting of $L$ layers.}
\end{figure}

Let $\vb{g}^{(k)} = \mqty(g_1^{(k)} & \dots & g_{m_k}^{(k)})^{\top}$ be the vector output of the $k$-th layer of the network. The layer $k$ performs a transformation on the output of the previous layer to compute
\begin{equation}
    \vb{g}^{(k)} = \sigma(\vb{W}^{(k)} \vb{g}^{(k-1)}) 
\end{equation}
where the $\ell$ component is
\begin{equation}
    g^{(k)}_\ell = \sigma\left( \displaystyle\sum_{\ell'=1}^{m_{k-1}} g^{(k-1)}_{\ell'}  w^{(k)}_{\ell,\ell'} \right) \hspace{2.5mm} 
    \begin{array}{l}
        \ell = 1, \hdots, m_k \\
     \ell' = 1, \hdots, m_{k-1} \\
    \end{array}
\end{equation}
where $\sigma(x)$ is the \emph{activation} function, \textit{e.g.} the \emph{Rectified Linear Unit} (ReLU):
\[
    \sigma(x) = \max \{ x, 0\}. 
\]

All these layers have trainable parameters, that get adjusted via an optimization algorithm like \emph{Stochastic Gradient Descent} to minimize a \emph{loss function}. These parameters are the weights
\[
    \vb{W}^{(1)}, \hdots, \vb{W}^{(L)}
\]
including the biases.

With this architecture, the network output is 
\begin{equation}
    \mathbf{g}^{\mathrm{out}} = \sigma\left( \vb{W}^{(L)} \left( \hdots \left( \mathbf{W}^{(2)} \sigma\left( \mathbf{W}^{(1)} \mathbf{f}^{\mathrm{in}}  \right) \right) \hdots \right) \right).
\end{equation}

In principle, this architecture is as general as it gets: deep feed-forward neural networks are \textit{provably} \emph{universal}, meaning that provided enough units, they can approximate any function with any desired accuracy. However, this comes with a price:
\begin{itemize}
    \item We can make them \emph{arbitrarily complex};
    
    \item The number of \emph{parameters} increases very rapidly and can get huge;
    
    \item The two points above combined make it so these network can become very difficult to \emph{optimize};
    
    \item Even then, with this architecture is very difficult to achieve \emph{generalization}, since often they become \textit{too powerful} and manage to represent perfectly the data, overfitting and losing generalization power.
\end{itemize}

\section{Need for Priors} 
%!TEX root = ../root.tex

\section{Entropy}

\paragraph{Intuition}

In information theory, the \emph{entropy} of a random variable is the average level of ``information'', or dually ``uncertainty'', that is inherent in the variable's possible outcomes.

Let's look at one such outcome, let's call it the event $E$, and say that it will occur with probability $p(E)$.

The intuition is that if the event $E$ is very likely to happen, then knowing that it will happen does not bring any interesting information. On the contrary, what is truly informative is knowing that something that happens very infrequently will indeed happen.

Therefore, the \emph{information content} (or \emph{surprisal}) carried by knowing that the event $E$ will happen can be quantified as a function that decreases with $p(E)$. In particular, this function is defined as:
\begin{equation}
	I(E) \triangleq -\log p(E) = \log \frac{1}{p(E)}.
\end{equation}

A random variable has a probability distribution defined over all the events that it encodes. Then, we can compute the average information inherent in the variable as the weighted average of the information carried by each one of its events, weighted by the probability that it will actually happen. 

A random variable in which all the outcomes are equally likely has high entropy, since there is maximum uncertainty about its outcome. A random variable in which there are certain outcomes that are much more likely than others has low entropy, since there is less uncertainty about its outcome.

A random variable that has \emph{more} outcomes than another has \emph{higher} entropy, since there is more uncertainty about its outcome.

\paragraph{Definition}

Let $X$ be a random variable, with possible values $\{ x_1, \dots, x_n \}$. Let $P(X)$ be the probability mass function defining a probability distribution over all the possible values of $X$.

Then, we call \emph{entropy} the average information represented in the distribution:
\begin{equation}
	H(X) \triangleq - \sum_{i=1}^{n} p(x_i) \log p(x_i),
\end{equation}
where $p(x_i) \equiv P(X = x_i)$.

\section{Kullback-Leibler divergence}

\paragraph{Intuition}

Let's consider two distributions of probability $p$ and $q$. Usually, $p$ represents the data, while $q$ a model, or in general an approximation of $p$, and we want to know how good of an approximation this is.

Then the Kullback-Leibler divergence is interpreted as the average loss of information content that we have when representing samples of $p$ (the data) using an \emph{optimal code} (something that does not introduce additional uncertainty beside the one intrinsic to the distribution) for $q$ (the model) instead of an optimal code for $p$.

\paragraph{Definition}
Let $p$ and $q$ be two probability distributions defined over the same space $X$. 

Then we call Kullback-Leibler divergence the measure
\begin{equation}
    KL (p \| q) \triangleq \sum_{x \in X} p(x) \log \left( \frac{p(x)}{q(x)} \right)
\end{equation}
which is equivalent to
\begin{equation}
    KL (p \| q) = -\sum_{x \in X} p(x) \log \left( \frac{q(x)}{p(x)} \right).
\end{equation}

\paragraph{Properties}

The KL-divergence:
\begin{itemize}
    \item is \emph{not} symmetric.
    \item is \emph{not} a distance, since it is not symmetric.
    \item is always non-negative.
    \item can be expressed as an expectation.
    \begin{equation}
        \begin{aligned}
            KL (P \| Q) & = \sum_{x \in X} p(x) \log \left( \frac{p(x)}{q(x)} \right) \\
            & = \mathbb{E}_{x \sim p(x)} \log \left( \frac{p(x)}{q(x)} \right).
        \end{aligned}
    \end{equation}
    \item measures the dissimilarity of two distributions in terms of their entropy.
    \begin{equation}\label{eq:KL-divergence}
        KL(p \| q) \approx H(q) - H(p)
    \end{equation} 
    
    In fact, Substituting the definition of entropy in \cref{eq:KL-divergence} we go back to the definition.
    \begin{align}
        KL(p \| q) &\approx - \sum_{x}q(x)\log \ q(x) + \sum_{x}p(x)\log \ p(x) \\
        KL(p \| q) &= - \sum_{x}p(x)\log \ q(x) + \sum_{x}p(x)\log \ p(x) \\
        KL(p \| q) &= \sum_{x}p(x)\log \frac{p(x)}{q(x)}
    \end{align}
\end{itemize}

\section{Convolution} 
%!TEX root = ../root.tex

\section{Entropy}

\paragraph{Intuition}

In information theory, the \emph{entropy} of a random variable is the average level of ``information'', or dually ``uncertainty'', that is inherent in the variable's possible outcomes.

Let's look at one such outcome, let's call it the event $E$, and say that it will occur with probability $p(E)$.

The intuition is that if the event $E$ is very likely to happen, then knowing that it will happen does not bring any interesting information. On the contrary, what is truly informative is knowing that something that happens very infrequently will indeed happen.

Therefore, the \emph{information content} (or \emph{surprisal}) carried by knowing that the event $E$ will happen can be quantified as a function that decreases with $p(E)$. In particular, this function is defined as:
\begin{equation}
	I(E) \triangleq -\log p(E) = \log \frac{1}{p(E)}.
\end{equation}

A random variable has a probability distribution defined over all the events that it encodes. Then, we can compute the average information inherent in the variable as the weighted average of the information carried by each one of its events, weighted by the probability that it will actually happen. 

A random variable in which all the outcomes are equally likely has high entropy, since there is maximum uncertainty about its outcome. A random variable in which there are certain outcomes that are much more likely than others has low entropy, since there is less uncertainty about its outcome.

A random variable that has \emph{more} outcomes than another has \emph{higher} entropy, since there is more uncertainty about its outcome.

\paragraph{Definition}

Let $X$ be a random variable, with possible values $\{ x_1, \dots, x_n \}$. Let $P(X)$ be the probability mass function defining a probability distribution over all the possible values of $X$.

Then, we call \emph{entropy} the average information represented in the distribution:
\begin{equation}
	H(X) \triangleq - \sum_{i=1}^{n} p(x_i) \log p(x_i),
\end{equation}
where $p(x_i) \equiv P(X = x_i)$.

\section{Kullback-Leibler divergence}

\paragraph{Intuition}

Let's consider two distributions of probability $p$ and $q$. Usually, $p$ represents the data, while $q$ a model, or in general an approximation of $p$, and we want to know how good of an approximation this is.

Then the Kullback-Leibler divergence is interpreted as the average loss of information content that we have when representing samples of $p$ (the data) using an \emph{optimal code} (something that does not introduce additional uncertainty beside the one intrinsic to the distribution) for $q$ (the model) instead of an optimal code for $p$.

\paragraph{Definition}
Let $p$ and $q$ be two probability distributions defined over the same space $X$. 

Then we call Kullback-Leibler divergence the measure
\begin{equation}
    KL (p \| q) \triangleq \sum_{x \in X} p(x) \log \left( \frac{p(x)}{q(x)} \right)
\end{equation}
which is equivalent to
\begin{equation}
    KL (p \| q) = -\sum_{x \in X} p(x) \log \left( \frac{q(x)}{p(x)} \right).
\end{equation}

\paragraph{Properties}

The KL-divergence:
\begin{itemize}
    \item is \emph{not} symmetric.
    \item is \emph{not} a distance, since it is not symmetric.
    \item is always non-negative.
    \item can be expressed as an expectation.
    \begin{equation}
        \begin{aligned}
            KL (P \| Q) & = \sum_{x \in X} p(x) \log \left( \frac{p(x)}{q(x)} \right) \\
            & = \mathbb{E}_{x \sim p(x)} \log \left( \frac{p(x)}{q(x)} \right).
        \end{aligned}
    \end{equation}
    \item measures the dissimilarity of two distributions in terms of their entropy.
    \begin{equation}\label{eq:KL-divergence}
        KL(p \| q) \approx H(q) - H(p)
    \end{equation} 
    
    In fact, Substituting the definition of entropy in \cref{eq:KL-divergence} we go back to the definition.
    \begin{align}
        KL(p \| q) &\approx - \sum_{x}q(x)\log \ q(x) + \sum_{x}p(x)\log \ p(x) \\
        KL(p \| q) &= - \sum_{x}p(x)\log \ q(x) + \sum_{x}p(x)\log \ p(x) \\
        KL(p \| q) &= \sum_{x}p(x)\log \frac{p(x)}{q(x)}
    \end{align}
\end{itemize}

\section{Convolutional Neural Networks} 
%!TEX root = ../root.tex

\section{Entropy}

\paragraph{Intuition}

In information theory, the \emph{entropy} of a random variable is the average level of ``information'', or dually ``uncertainty'', that is inherent in the variable's possible outcomes.

Let's look at one such outcome, let's call it the event $E$, and say that it will occur with probability $p(E)$.

The intuition is that if the event $E$ is very likely to happen, then knowing that it will happen does not bring any interesting information. On the contrary, what is truly informative is knowing that something that happens very infrequently will indeed happen.

Therefore, the \emph{information content} (or \emph{surprisal}) carried by knowing that the event $E$ will happen can be quantified as a function that decreases with $p(E)$. In particular, this function is defined as:
\begin{equation}
	I(E) \triangleq -\log p(E) = \log \frac{1}{p(E)}.
\end{equation}

A random variable has a probability distribution defined over all the events that it encodes. Then, we can compute the average information inherent in the variable as the weighted average of the information carried by each one of its events, weighted by the probability that it will actually happen. 

A random variable in which all the outcomes are equally likely has high entropy, since there is maximum uncertainty about its outcome. A random variable in which there are certain outcomes that are much more likely than others has low entropy, since there is less uncertainty about its outcome.

A random variable that has \emph{more} outcomes than another has \emph{higher} entropy, since there is more uncertainty about its outcome.

\paragraph{Definition}

Let $X$ be a random variable, with possible values $\{ x_1, \dots, x_n \}$. Let $P(X)$ be the probability mass function defining a probability distribution over all the possible values of $X$.

Then, we call \emph{entropy} the average information represented in the distribution:
\begin{equation}
	H(X) \triangleq - \sum_{i=1}^{n} p(x_i) \log p(x_i),
\end{equation}
where $p(x_i) \equiv P(X = x_i)$.

\section{Kullback-Leibler divergence}

\paragraph{Intuition}

Let's consider two distributions of probability $p$ and $q$. Usually, $p$ represents the data, while $q$ a model, or in general an approximation of $p$, and we want to know how good of an approximation this is.

Then the Kullback-Leibler divergence is interpreted as the average loss of information content that we have when representing samples of $p$ (the data) using an \emph{optimal code} (something that does not introduce additional uncertainty beside the one intrinsic to the distribution) for $q$ (the model) instead of an optimal code for $p$.

\paragraph{Definition}
Let $p$ and $q$ be two probability distributions defined over the same space $X$. 

Then we call Kullback-Leibler divergence the measure
\begin{equation}
    KL (p \| q) \triangleq \sum_{x \in X} p(x) \log \left( \frac{p(x)}{q(x)} \right)
\end{equation}
which is equivalent to
\begin{equation}
    KL (p \| q) = -\sum_{x \in X} p(x) \log \left( \frac{q(x)}{p(x)} \right).
\end{equation}

\paragraph{Properties}

The KL-divergence:
\begin{itemize}
    \item is \emph{not} symmetric.
    \item is \emph{not} a distance, since it is not symmetric.
    \item is always non-negative.
    \item can be expressed as an expectation.
    \begin{equation}
        \begin{aligned}
            KL (P \| Q) & = \sum_{x \in X} p(x) \log \left( \frac{p(x)}{q(x)} \right) \\
            & = \mathbb{E}_{x \sim p(x)} \log \left( \frac{p(x)}{q(x)} \right).
        \end{aligned}
    \end{equation}
    \item measures the dissimilarity of two distributions in terms of their entropy.
    \begin{equation}\label{eq:KL-divergence}
        KL(p \| q) \approx H(q) - H(p)
    \end{equation} 
    
    In fact, Substituting the definition of entropy in \cref{eq:KL-divergence} we go back to the definition.
    \begin{align}
        KL(p \| q) &\approx - \sum_{x}q(x)\log \ q(x) + \sum_{x}p(x)\log \ p(x) \\
        KL(p \| q) &= - \sum_{x}p(x)\log \ q(x) + \sum_{x}p(x)\log \ p(x) \\
        KL(p \| q) &= \sum_{x}p(x)\log \frac{p(x)}{q(x)}
    \end{align}
\end{itemize}

