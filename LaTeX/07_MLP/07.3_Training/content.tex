%!TEX root=../../root.tex

Given a MLP with training set $\mathcal{D} = \{ \mathbf{x}_i, \mathbf{y}_i \}_{i=1}^{n}$
\begin{equation} 
    g_{\bm{\Theta}}(\mathbf{x}_i)= (\sigma \circ f_{\bm{\Theta}_n})  \circ (\sigma \circ f_{\bm{\Theta}_{n-1}}) \circ \cdots \circ (\sigma \circ  f_{\bm{\Theta}_1})(\mathbf{x}_i) = \mathbf{y}_i  
\end{equation}
and a suitable loss function, such as the Mean Squared Error loss defined as
\begin{equation} 
	\ell_{\bm{\Theta}} ( \{ \mathbf{x}_i, \mathbf{y}_i\}) = \frac{1}{n}\sum_{i=1}^n \| \mathbf{y}_i - g_{\bm{\Theta}}(\mathbf{x}_i) \|_2^2,
\end{equation}
we have seen that solving for the weights $\bm{\Theta}$ is referred to as \emph{training}.

In general, the loss will be a \emph{non-convex} function of the weights $\bm{\Theta}$. As we have seen, the following \emph{special cases} are convex:
\begin{itemize}
	\item One layer, no activation, MSE loss ($\Rightarrow$ linear regression).
	\item One layer, sigmoid activation, logistic loss ($\Rightarrow$ logistic regression).
\end{itemize}


We have also seen that training is usually performed using gradient descent-like algorithms, that require the computation of gradients $\nabla \ell_{\bm{\Theta}}$. For the basic MSE, this means
\begin{align}
    \nabla \ell_{\bm{\Theta}} ( \{ \mathbf{x}_i, \mathbf{y}_i\}) & =
    \frac{1}{n}  \sum_{i=1}^n \nabla_{\bm{\Theta}} \| \mathbf{y}_i - g_{\bm{\Theta}}(\mathbf{x}_i) \|_2^2  \\
    & = \frac{1}{n}\sum_{i=1}^n \nabla_{\bm{\Theta}} \| (\mathbf{y}_i -  (\sigma ( f_{\bm{\Theta}_n} (\sigma( f_{\bm{\Theta}_{n-1}}( \cdots (\sigma (  f_{\bm{\Theta}_1}(\mathbf{x}_i))\cdots ))))))) \|_2^2 
\end{align}

Of course for different loss functions and different models we will have different expressions for the gradient; however, computing the gradients \emph{by hand} would be infeasible.

The derivatives could be approximated numerically by approximating the limit of the ratio, but doing so requires $O(\#\textrm{weights})$ evaluations of $\ell_{\bm{\Theta}}$. The gradient can be computed automatically using the \emph{chain rule}, but this is still sub-optimal. What we want to do is automatize this \emph{computational step} efficiently, and this is done by \emph{automatic differentiation}.

\subsection{Computational graphs} 
%!TEX root=../../../root.tex

Consider a linear map $T:V \to W$, a basis $v_1,\dots,v_n \in V$ and a basis $w_1,\dots,w_m\in W$.
{
The \emph{matrix} of $T$ in these bases is the $m\times n$ array of values in $\mathbb{R}$
%
\[
\mathbf{T} = \begin{pmatrix}
    T_{1,1} & \cdots & T_{1,n} \\
    \vdots & & \vdots\\
    T_{m,1} & \cdots & T_{m,n}
  \end{pmatrix}
\]
%
whose entries $T_{i,j}$ are defined by
%
\[
{\color{darkgreen}Tv_j} = {\color{red}T_{1,j}} w_1 + \cdots + {\color{red}T_{m,j}} w_m
\]
}%

{
Hence each column of $\mathbf{T}$ contains the {\color{red}linear combination coefficients} for the {\color{darkgreen}image via $T$ of a basis vector from $V$}
}

{
In other words, the matrix encodes {\color{darkgreen}how basis vectors are mapped}, and this is enough to map all other vectors in their span, since:
\[ Tv = T ( \sum_j \alpha_j v_j ) = \sum_j T(\alpha_j v_j) = \sum_j \alpha_j {\color{darkgreen}Tv_j} \]
}

The matrix is a \emph{representation} for a linear map, and it \emph{depends on the choice of bases}.
Suppose $v \in V$ is an arbitrary vector, while $v_1,\dots,v_n$ is a basis of $V$. The matrix of $v$ wrt this basis is the $n\times 1$ matrix:
%
\[
\mathbf{v} = \begin{pmatrix}
    c_1 \\
    \vdots\\
    c_n
  \end{pmatrix}
\]
%
so that
%
\[
v = c_1 v_1 + \cdots c_n v_n
\]

Once again, we see that the matrix \emph{depends on the choice of basis} for $V$


\begin{itemize}
\item \textbf{addition}:  the matrix of $S+T$ can be obtained by summing the matrices of $S$ and $T${; this only makes sense if the \emph{same bases} are used for $S$, $T$, and $S+T$}
\item \textbf{scalar multiplication}: given $\lambda\in\mathbb{R}$, the matrix for $\lambda T$ is given by $\lambda$ times the matrix of $T$
\end{itemize}

{
In fact, we have just shown that \emph{matrices form a vector space} (Q1: what is the additive identity?) {(Q2: what is the vector space dimension?)}
}

{
We call $\mathbb{R}^{m\times n}$ the vector space of all $m\times n$ matrices with values in $\mathbb{R}$
}

{
\begin{itemize}
\item \textbf{product}: the matrix for $ST$ can be computed by the \emph{matrix product} between $\mathbf{S}$ and $\mathbf{T}$; in fact, the matrix product is defined precisely to make this work

{
Q3: is matrix product commutative?
}

{
Q4: do we need the same bases for $S:U\to V$ and $T:V \to W$?
}

\end{itemize}
}



Consider a linear map $T:V \to W$, a basis $v_1,\dots,v_n \in V$ and a basis $w_1,\dots,w_m\in W$.

%\medskip
%The $k$-th column of $\mathbf{T}$ equals the matrix vector $\mathbf{v}_k$:

%figure

From the definition of matrix product, one can show that it operates on a vector matrix as expected:
\[
\mathbf{Tv} =\mathbf{w} \quad \Leftrightarrow \quad Tv=w
\]
where $\mathbf{Tv}$ is the matrix product of $\mathbf{T}$ and $\mathbf{v}$, while $Tv$ simply denotes the function evaluation $T(v)$


{
\textbf{Remember:} $\mathbf{T}, \mathbf{v}, \mathbf{w}$ must follow a coherent choice of bases in order for the above to make sense. $\mathbf{v}$ can not be expressed in basis $\color{red}(\tilde{v}_1,\dots,\tilde{v}_n)$ if $\mathbf{T}$ only knows how to map basis vectors $\color{blue}({v}_1,\dots,{v}_n)$.
%
\[
T{\color{blue}v_j} = {\color{blue}T_{1,j}} w_1 + \cdots + {\color{blue}T_{m,j}} w_m
\]
%
%\[ Tv =  \sum_j \alpha_j T{\color{red}\tilde{v}_j} \]
}


\[
\underbrace{
\begin{pmatrix}
    T_{1,1} & \cdots & T_{1,n} \\
    \vdots & & \vdots\\
    T_{m,1} & \cdots & T_{m,n}
  \end{pmatrix}}_{\mathbf{T}}
  %
  \underbrace{
  \begin{pmatrix}
    c_1 \\
    \vdots \\
    c_n
  \end{pmatrix}}_{\mathbf{c}} =
  %
   \sum_{j=1}^n c_j \hspace{-0.6cm}
   \underbrace{\begin{pmatrix}
    {\color{red}T_{1,j}}  \\
    \vdots \\
    {\color{red}T_{m,j}}
  \end{pmatrix}}_{\mathrm{Tv_j~wrt~(w_1,\dots,w_m)}}
\]
%
\smallskip

Because recall that, for bases $v_1,\dots,v_n \in V$ and $w_1,\dots,w_m\in W$:
%
\[
Tv_j = {\color{red}T_{1,j}} w_1 + \cdots + {\color{red}T_{m,j}} w_m
\]

{
We see then that vector $c=\sum_j c_j v_j$ is mapped to $Tc = \sum_j c_j Tv_j$.

In other words, matrix product is behaving as expected.
}









\subsection{Automatic differentiation: forward mode} 
%!TEX root=../../../root.tex

Consider a linear map $T:V \to W$, a basis $v_1,\dots,v_n \in V$ and a basis $w_1,\dots,w_m\in W$.
{
The \emph{matrix} of $T$ in these bases is the $m\times n$ array of values in $\mathbb{R}$
%
\[
\mathbf{T} = \begin{pmatrix}
    T_{1,1} & \cdots & T_{1,n} \\
    \vdots & & \vdots\\
    T_{m,1} & \cdots & T_{m,n}
  \end{pmatrix}
\]
%
whose entries $T_{i,j}$ are defined by
%
\[
{\color{darkgreen}Tv_j} = {\color{red}T_{1,j}} w_1 + \cdots + {\color{red}T_{m,j}} w_m
\]
}%

{
Hence each column of $\mathbf{T}$ contains the {\color{red}linear combination coefficients} for the {\color{darkgreen}image via $T$ of a basis vector from $V$}
}

{
In other words, the matrix encodes {\color{darkgreen}how basis vectors are mapped}, and this is enough to map all other vectors in their span, since:
\[ Tv = T ( \sum_j \alpha_j v_j ) = \sum_j T(\alpha_j v_j) = \sum_j \alpha_j {\color{darkgreen}Tv_j} \]
}

The matrix is a \emph{representation} for a linear map, and it \emph{depends on the choice of bases}.
Suppose $v \in V$ is an arbitrary vector, while $v_1,\dots,v_n$ is a basis of $V$. The matrix of $v$ wrt this basis is the $n\times 1$ matrix:
%
\[
\mathbf{v} = \begin{pmatrix}
    c_1 \\
    \vdots\\
    c_n
  \end{pmatrix}
\]
%
so that
%
\[
v = c_1 v_1 + \cdots c_n v_n
\]

Once again, we see that the matrix \emph{depends on the choice of basis} for $V$


\begin{itemize}
\item \textbf{addition}:  the matrix of $S+T$ can be obtained by summing the matrices of $S$ and $T${; this only makes sense if the \emph{same bases} are used for $S$, $T$, and $S+T$}
\item \textbf{scalar multiplication}: given $\lambda\in\mathbb{R}$, the matrix for $\lambda T$ is given by $\lambda$ times the matrix of $T$
\end{itemize}

{
In fact, we have just shown that \emph{matrices form a vector space} (Q1: what is the additive identity?) {(Q2: what is the vector space dimension?)}
}

{
We call $\mathbb{R}^{m\times n}$ the vector space of all $m\times n$ matrices with values in $\mathbb{R}$
}

{
\begin{itemize}
\item \textbf{product}: the matrix for $ST$ can be computed by the \emph{matrix product} between $\mathbf{S}$ and $\mathbf{T}$; in fact, the matrix product is defined precisely to make this work

{
Q3: is matrix product commutative?
}

{
Q4: do we need the same bases for $S:U\to V$ and $T:V \to W$?
}

\end{itemize}
}



Consider a linear map $T:V \to W$, a basis $v_1,\dots,v_n \in V$ and a basis $w_1,\dots,w_m\in W$.

%\medskip
%The $k$-th column of $\mathbf{T}$ equals the matrix vector $\mathbf{v}_k$:

%figure

From the definition of matrix product, one can show that it operates on a vector matrix as expected:
\[
\mathbf{Tv} =\mathbf{w} \quad \Leftrightarrow \quad Tv=w
\]
where $\mathbf{Tv}$ is the matrix product of $\mathbf{T}$ and $\mathbf{v}$, while $Tv$ simply denotes the function evaluation $T(v)$


{
\textbf{Remember:} $\mathbf{T}, \mathbf{v}, \mathbf{w}$ must follow a coherent choice of bases in order for the above to make sense. $\mathbf{v}$ can not be expressed in basis $\color{red}(\tilde{v}_1,\dots,\tilde{v}_n)$ if $\mathbf{T}$ only knows how to map basis vectors $\color{blue}({v}_1,\dots,{v}_n)$.
%
\[
T{\color{blue}v_j} = {\color{blue}T_{1,j}} w_1 + \cdots + {\color{blue}T_{m,j}} w_m
\]
%
%\[ Tv =  \sum_j \alpha_j T{\color{red}\tilde{v}_j} \]
}


\[
\underbrace{
\begin{pmatrix}
    T_{1,1} & \cdots & T_{1,n} \\
    \vdots & & \vdots\\
    T_{m,1} & \cdots & T_{m,n}
  \end{pmatrix}}_{\mathbf{T}}
  %
  \underbrace{
  \begin{pmatrix}
    c_1 \\
    \vdots \\
    c_n
  \end{pmatrix}}_{\mathbf{c}} =
  %
   \sum_{j=1}^n c_j \hspace{-0.6cm}
   \underbrace{\begin{pmatrix}
    {\color{red}T_{1,j}}  \\
    \vdots \\
    {\color{red}T_{m,j}}
  \end{pmatrix}}_{\mathrm{Tv_j~wrt~(w_1,\dots,w_m)}}
\]
%
\smallskip

Because recall that, for bases $v_1,\dots,v_n \in V$ and $w_1,\dots,w_m\in W$:
%
\[
Tv_j = {\color{red}T_{1,j}} w_1 + \cdots + {\color{red}T_{m,j}} w_m
\]

{
We see then that vector $c=\sum_j c_j v_j$ is mapped to $Tc = \sum_j c_j Tv_j$.

In other words, matrix product is behaving as expected.
}









\subsection{Automatic differentiation: reverse mode} 
%!TEX root=../../../root.tex

Consider a linear map $T:V \to W$, a basis $v_1,\dots,v_n \in V$ and a basis $w_1,\dots,w_m\in W$.
{
The \emph{matrix} of $T$ in these bases is the $m\times n$ array of values in $\mathbb{R}$
%
\[
\mathbf{T} = \begin{pmatrix}
    T_{1,1} & \cdots & T_{1,n} \\
    \vdots & & \vdots\\
    T_{m,1} & \cdots & T_{m,n}
  \end{pmatrix}
\]
%
whose entries $T_{i,j}$ are defined by
%
\[
{\color{darkgreen}Tv_j} = {\color{red}T_{1,j}} w_1 + \cdots + {\color{red}T_{m,j}} w_m
\]
}%

{
Hence each column of $\mathbf{T}$ contains the {\color{red}linear combination coefficients} for the {\color{darkgreen}image via $T$ of a basis vector from $V$}
}

{
In other words, the matrix encodes {\color{darkgreen}how basis vectors are mapped}, and this is enough to map all other vectors in their span, since:
\[ Tv = T ( \sum_j \alpha_j v_j ) = \sum_j T(\alpha_j v_j) = \sum_j \alpha_j {\color{darkgreen}Tv_j} \]
}

The matrix is a \emph{representation} for a linear map, and it \emph{depends on the choice of bases}.
Suppose $v \in V$ is an arbitrary vector, while $v_1,\dots,v_n$ is a basis of $V$. The matrix of $v$ wrt this basis is the $n\times 1$ matrix:
%
\[
\mathbf{v} = \begin{pmatrix}
    c_1 \\
    \vdots\\
    c_n
  \end{pmatrix}
\]
%
so that
%
\[
v = c_1 v_1 + \cdots c_n v_n
\]

Once again, we see that the matrix \emph{depends on the choice of basis} for $V$


\begin{itemize}
\item \textbf{addition}:  the matrix of $S+T$ can be obtained by summing the matrices of $S$ and $T${; this only makes sense if the \emph{same bases} are used for $S$, $T$, and $S+T$}
\item \textbf{scalar multiplication}: given $\lambda\in\mathbb{R}$, the matrix for $\lambda T$ is given by $\lambda$ times the matrix of $T$
\end{itemize}

{
In fact, we have just shown that \emph{matrices form a vector space} (Q1: what is the additive identity?) {(Q2: what is the vector space dimension?)}
}

{
We call $\mathbb{R}^{m\times n}$ the vector space of all $m\times n$ matrices with values in $\mathbb{R}$
}

{
\begin{itemize}
\item \textbf{product}: the matrix for $ST$ can be computed by the \emph{matrix product} between $\mathbf{S}$ and $\mathbf{T}$; in fact, the matrix product is defined precisely to make this work

{
Q3: is matrix product commutative?
}

{
Q4: do we need the same bases for $S:U\to V$ and $T:V \to W$?
}

\end{itemize}
}



Consider a linear map $T:V \to W$, a basis $v_1,\dots,v_n \in V$ and a basis $w_1,\dots,w_m\in W$.

%\medskip
%The $k$-th column of $\mathbf{T}$ equals the matrix vector $\mathbf{v}_k$:

%figure

From the definition of matrix product, one can show that it operates on a vector matrix as expected:
\[
\mathbf{Tv} =\mathbf{w} \quad \Leftrightarrow \quad Tv=w
\]
where $\mathbf{Tv}$ is the matrix product of $\mathbf{T}$ and $\mathbf{v}$, while $Tv$ simply denotes the function evaluation $T(v)$


{
\textbf{Remember:} $\mathbf{T}, \mathbf{v}, \mathbf{w}$ must follow a coherent choice of bases in order for the above to make sense. $\mathbf{v}$ can not be expressed in basis $\color{red}(\tilde{v}_1,\dots,\tilde{v}_n)$ if $\mathbf{T}$ only knows how to map basis vectors $\color{blue}({v}_1,\dots,{v}_n)$.
%
\[
T{\color{blue}v_j} = {\color{blue}T_{1,j}} w_1 + \cdots + {\color{blue}T_{m,j}} w_m
\]
%
%\[ Tv =  \sum_j \alpha_j T{\color{red}\tilde{v}_j} \]
}


\[
\underbrace{
\begin{pmatrix}
    T_{1,1} & \cdots & T_{1,n} \\
    \vdots & & \vdots\\
    T_{m,1} & \cdots & T_{m,n}
  \end{pmatrix}}_{\mathbf{T}}
  %
  \underbrace{
  \begin{pmatrix}
    c_1 \\
    \vdots \\
    c_n
  \end{pmatrix}}_{\mathbf{c}} =
  %
   \sum_{j=1}^n c_j \hspace{-0.6cm}
   \underbrace{\begin{pmatrix}
    {\color{red}T_{1,j}}  \\
    \vdots \\
    {\color{red}T_{m,j}}
  \end{pmatrix}}_{\mathrm{Tv_j~wrt~(w_1,\dots,w_m)}}
\]
%
\smallskip

Because recall that, for bases $v_1,\dots,v_n \in V$ and $w_1,\dots,w_m\in W$:
%
\[
Tv_j = {\color{red}T_{1,j}} w_1 + \cdots + {\color{red}T_{m,j}} w_m
\]

{
We see then that vector $c=\sum_j c_j v_j$ is mapped to $Tc = \sum_j c_j Tv_j$.

In other words, matrix product is behaving as expected.
}









\subsection{Automatic differentiation: complexity} 
%!TEX root=../../../root.tex

Consider a linear map $T:V \to W$, a basis $v_1,\dots,v_n \in V$ and a basis $w_1,\dots,w_m\in W$.
{
The \emph{matrix} of $T$ in these bases is the $m\times n$ array of values in $\mathbb{R}$
%
\[
\mathbf{T} = \begin{pmatrix}
    T_{1,1} & \cdots & T_{1,n} \\
    \vdots & & \vdots\\
    T_{m,1} & \cdots & T_{m,n}
  \end{pmatrix}
\]
%
whose entries $T_{i,j}$ are defined by
%
\[
{\color{darkgreen}Tv_j} = {\color{red}T_{1,j}} w_1 + \cdots + {\color{red}T_{m,j}} w_m
\]
}%

{
Hence each column of $\mathbf{T}$ contains the {\color{red}linear combination coefficients} for the {\color{darkgreen}image via $T$ of a basis vector from $V$}
}

{
In other words, the matrix encodes {\color{darkgreen}how basis vectors are mapped}, and this is enough to map all other vectors in their span, since:
\[ Tv = T ( \sum_j \alpha_j v_j ) = \sum_j T(\alpha_j v_j) = \sum_j \alpha_j {\color{darkgreen}Tv_j} \]
}

The matrix is a \emph{representation} for a linear map, and it \emph{depends on the choice of bases}.
Suppose $v \in V$ is an arbitrary vector, while $v_1,\dots,v_n$ is a basis of $V$. The matrix of $v$ wrt this basis is the $n\times 1$ matrix:
%
\[
\mathbf{v} = \begin{pmatrix}
    c_1 \\
    \vdots\\
    c_n
  \end{pmatrix}
\]
%
so that
%
\[
v = c_1 v_1 + \cdots c_n v_n
\]

Once again, we see that the matrix \emph{depends on the choice of basis} for $V$


\begin{itemize}
\item \textbf{addition}:  the matrix of $S+T$ can be obtained by summing the matrices of $S$ and $T${; this only makes sense if the \emph{same bases} are used for $S$, $T$, and $S+T$}
\item \textbf{scalar multiplication}: given $\lambda\in\mathbb{R}$, the matrix for $\lambda T$ is given by $\lambda$ times the matrix of $T$
\end{itemize}

{
In fact, we have just shown that \emph{matrices form a vector space} (Q1: what is the additive identity?) {(Q2: what is the vector space dimension?)}
}

{
We call $\mathbb{R}^{m\times n}$ the vector space of all $m\times n$ matrices with values in $\mathbb{R}$
}

{
\begin{itemize}
\item \textbf{product}: the matrix for $ST$ can be computed by the \emph{matrix product} between $\mathbf{S}$ and $\mathbf{T}$; in fact, the matrix product is defined precisely to make this work

{
Q3: is matrix product commutative?
}

{
Q4: do we need the same bases for $S:U\to V$ and $T:V \to W$?
}

\end{itemize}
}



Consider a linear map $T:V \to W$, a basis $v_1,\dots,v_n \in V$ and a basis $w_1,\dots,w_m\in W$.

%\medskip
%The $k$-th column of $\mathbf{T}$ equals the matrix vector $\mathbf{v}_k$:

%figure

From the definition of matrix product, one can show that it operates on a vector matrix as expected:
\[
\mathbf{Tv} =\mathbf{w} \quad \Leftrightarrow \quad Tv=w
\]
where $\mathbf{Tv}$ is the matrix product of $\mathbf{T}$ and $\mathbf{v}$, while $Tv$ simply denotes the function evaluation $T(v)$


{
\textbf{Remember:} $\mathbf{T}, \mathbf{v}, \mathbf{w}$ must follow a coherent choice of bases in order for the above to make sense. $\mathbf{v}$ can not be expressed in basis $\color{red}(\tilde{v}_1,\dots,\tilde{v}_n)$ if $\mathbf{T}$ only knows how to map basis vectors $\color{blue}({v}_1,\dots,{v}_n)$.
%
\[
T{\color{blue}v_j} = {\color{blue}T_{1,j}} w_1 + \cdots + {\color{blue}T_{m,j}} w_m
\]
%
%\[ Tv =  \sum_j \alpha_j T{\color{red}\tilde{v}_j} \]
}


\[
\underbrace{
\begin{pmatrix}
    T_{1,1} & \cdots & T_{1,n} \\
    \vdots & & \vdots\\
    T_{m,1} & \cdots & T_{m,n}
  \end{pmatrix}}_{\mathbf{T}}
  %
  \underbrace{
  \begin{pmatrix}
    c_1 \\
    \vdots \\
    c_n
  \end{pmatrix}}_{\mathbf{c}} =
  %
   \sum_{j=1}^n c_j \hspace{-0.6cm}
   \underbrace{\begin{pmatrix}
    {\color{red}T_{1,j}}  \\
    \vdots \\
    {\color{red}T_{m,j}}
  \end{pmatrix}}_{\mathrm{Tv_j~wrt~(w_1,\dots,w_m)}}
\]
%
\smallskip

Because recall that, for bases $v_1,\dots,v_n \in V$ and $w_1,\dots,w_m\in W$:
%
\[
Tv_j = {\color{red}T_{1,j}} w_1 + \cdots + {\color{red}T_{m,j}} w_m
\]

{
We see then that vector $c=\sum_j c_j v_j$ is mapped to $Tc = \sum_j c_j Tv_j$.

In other words, matrix product is behaving as expected.
}









\subsection{Backpropagation} 
%!TEX root=../../../root.tex

Consider a linear map $T:V \to W$, a basis $v_1,\dots,v_n \in V$ and a basis $w_1,\dots,w_m\in W$.
{
The \emph{matrix} of $T$ in these bases is the $m\times n$ array of values in $\mathbb{R}$
%
\[
\mathbf{T} = \begin{pmatrix}
    T_{1,1} & \cdots & T_{1,n} \\
    \vdots & & \vdots\\
    T_{m,1} & \cdots & T_{m,n}
  \end{pmatrix}
\]
%
whose entries $T_{i,j}$ are defined by
%
\[
{\color{darkgreen}Tv_j} = {\color{red}T_{1,j}} w_1 + \cdots + {\color{red}T_{m,j}} w_m
\]
}%

{
Hence each column of $\mathbf{T}$ contains the {\color{red}linear combination coefficients} for the {\color{darkgreen}image via $T$ of a basis vector from $V$}
}

{
In other words, the matrix encodes {\color{darkgreen}how basis vectors are mapped}, and this is enough to map all other vectors in their span, since:
\[ Tv = T ( \sum_j \alpha_j v_j ) = \sum_j T(\alpha_j v_j) = \sum_j \alpha_j {\color{darkgreen}Tv_j} \]
}

The matrix is a \emph{representation} for a linear map, and it \emph{depends on the choice of bases}.
Suppose $v \in V$ is an arbitrary vector, while $v_1,\dots,v_n$ is a basis of $V$. The matrix of $v$ wrt this basis is the $n\times 1$ matrix:
%
\[
\mathbf{v} = \begin{pmatrix}
    c_1 \\
    \vdots\\
    c_n
  \end{pmatrix}
\]
%
so that
%
\[
v = c_1 v_1 + \cdots c_n v_n
\]

Once again, we see that the matrix \emph{depends on the choice of basis} for $V$


\begin{itemize}
\item \textbf{addition}:  the matrix of $S+T$ can be obtained by summing the matrices of $S$ and $T${; this only makes sense if the \emph{same bases} are used for $S$, $T$, and $S+T$}
\item \textbf{scalar multiplication}: given $\lambda\in\mathbb{R}$, the matrix for $\lambda T$ is given by $\lambda$ times the matrix of $T$
\end{itemize}

{
In fact, we have just shown that \emph{matrices form a vector space} (Q1: what is the additive identity?) {(Q2: what is the vector space dimension?)}
}

{
We call $\mathbb{R}^{m\times n}$ the vector space of all $m\times n$ matrices with values in $\mathbb{R}$
}

{
\begin{itemize}
\item \textbf{product}: the matrix for $ST$ can be computed by the \emph{matrix product} between $\mathbf{S}$ and $\mathbf{T}$; in fact, the matrix product is defined precisely to make this work

{
Q3: is matrix product commutative?
}

{
Q4: do we need the same bases for $S:U\to V$ and $T:V \to W$?
}

\end{itemize}
}



Consider a linear map $T:V \to W$, a basis $v_1,\dots,v_n \in V$ and a basis $w_1,\dots,w_m\in W$.

%\medskip
%The $k$-th column of $\mathbf{T}$ equals the matrix vector $\mathbf{v}_k$:

%figure

From the definition of matrix product, one can show that it operates on a vector matrix as expected:
\[
\mathbf{Tv} =\mathbf{w} \quad \Leftrightarrow \quad Tv=w
\]
where $\mathbf{Tv}$ is the matrix product of $\mathbf{T}$ and $\mathbf{v}$, while $Tv$ simply denotes the function evaluation $T(v)$


{
\textbf{Remember:} $\mathbf{T}, \mathbf{v}, \mathbf{w}$ must follow a coherent choice of bases in order for the above to make sense. $\mathbf{v}$ can not be expressed in basis $\color{red}(\tilde{v}_1,\dots,\tilde{v}_n)$ if $\mathbf{T}$ only knows how to map basis vectors $\color{blue}({v}_1,\dots,{v}_n)$.
%
\[
T{\color{blue}v_j} = {\color{blue}T_{1,j}} w_1 + \cdots + {\color{blue}T_{m,j}} w_m
\]
%
%\[ Tv =  \sum_j \alpha_j T{\color{red}\tilde{v}_j} \]
}


\[
\underbrace{
\begin{pmatrix}
    T_{1,1} & \cdots & T_{1,n} \\
    \vdots & & \vdots\\
    T_{m,1} & \cdots & T_{m,n}
  \end{pmatrix}}_{\mathbf{T}}
  %
  \underbrace{
  \begin{pmatrix}
    c_1 \\
    \vdots \\
    c_n
  \end{pmatrix}}_{\mathbf{c}} =
  %
   \sum_{j=1}^n c_j \hspace{-0.6cm}
   \underbrace{\begin{pmatrix}
    {\color{red}T_{1,j}}  \\
    \vdots \\
    {\color{red}T_{m,j}}
  \end{pmatrix}}_{\mathrm{Tv_j~wrt~(w_1,\dots,w_m)}}
\]
%
\smallskip

Because recall that, for bases $v_1,\dots,v_n \in V$ and $w_1,\dots,w_m\in W$:
%
\[
Tv_j = {\color{red}T_{1,j}} w_1 + \cdots + {\color{red}T_{m,j}} w_m
\]

{
We see then that vector $c=\sum_j c_j v_j$ is mapped to $Tc = \sum_j c_j Tv_j$.

In other words, matrix product is behaving as expected.
}









\subsection{Observations} 
%!TEX root=../../../root.tex

Consider a linear map $T:V \to W$, a basis $v_1,\dots,v_n \in V$ and a basis $w_1,\dots,w_m\in W$.
{
The \emph{matrix} of $T$ in these bases is the $m\times n$ array of values in $\mathbb{R}$
%
\[
\mathbf{T} = \begin{pmatrix}
    T_{1,1} & \cdots & T_{1,n} \\
    \vdots & & \vdots\\
    T_{m,1} & \cdots & T_{m,n}
  \end{pmatrix}
\]
%
whose entries $T_{i,j}$ are defined by
%
\[
{\color{darkgreen}Tv_j} = {\color{red}T_{1,j}} w_1 + \cdots + {\color{red}T_{m,j}} w_m
\]
}%

{
Hence each column of $\mathbf{T}$ contains the {\color{red}linear combination coefficients} for the {\color{darkgreen}image via $T$ of a basis vector from $V$}
}

{
In other words, the matrix encodes {\color{darkgreen}how basis vectors are mapped}, and this is enough to map all other vectors in their span, since:
\[ Tv = T ( \sum_j \alpha_j v_j ) = \sum_j T(\alpha_j v_j) = \sum_j \alpha_j {\color{darkgreen}Tv_j} \]
}

The matrix is a \emph{representation} for a linear map, and it \emph{depends on the choice of bases}.
Suppose $v \in V$ is an arbitrary vector, while $v_1,\dots,v_n$ is a basis of $V$. The matrix of $v$ wrt this basis is the $n\times 1$ matrix:
%
\[
\mathbf{v} = \begin{pmatrix}
    c_1 \\
    \vdots\\
    c_n
  \end{pmatrix}
\]
%
so that
%
\[
v = c_1 v_1 + \cdots c_n v_n
\]

Once again, we see that the matrix \emph{depends on the choice of basis} for $V$


\begin{itemize}
\item \textbf{addition}:  the matrix of $S+T$ can be obtained by summing the matrices of $S$ and $T${; this only makes sense if the \emph{same bases} are used for $S$, $T$, and $S+T$}
\item \textbf{scalar multiplication}: given $\lambda\in\mathbb{R}$, the matrix for $\lambda T$ is given by $\lambda$ times the matrix of $T$
\end{itemize}

{
In fact, we have just shown that \emph{matrices form a vector space} (Q1: what is the additive identity?) {(Q2: what is the vector space dimension?)}
}

{
We call $\mathbb{R}^{m\times n}$ the vector space of all $m\times n$ matrices with values in $\mathbb{R}$
}

{
\begin{itemize}
\item \textbf{product}: the matrix for $ST$ can be computed by the \emph{matrix product} between $\mathbf{S}$ and $\mathbf{T}$; in fact, the matrix product is defined precisely to make this work

{
Q3: is matrix product commutative?
}

{
Q4: do we need the same bases for $S:U\to V$ and $T:V \to W$?
}

\end{itemize}
}



Consider a linear map $T:V \to W$, a basis $v_1,\dots,v_n \in V$ and a basis $w_1,\dots,w_m\in W$.

%\medskip
%The $k$-th column of $\mathbf{T}$ equals the matrix vector $\mathbf{v}_k$:

%figure

From the definition of matrix product, one can show that it operates on a vector matrix as expected:
\[
\mathbf{Tv} =\mathbf{w} \quad \Leftrightarrow \quad Tv=w
\]
where $\mathbf{Tv}$ is the matrix product of $\mathbf{T}$ and $\mathbf{v}$, while $Tv$ simply denotes the function evaluation $T(v)$


{
\textbf{Remember:} $\mathbf{T}, \mathbf{v}, \mathbf{w}$ must follow a coherent choice of bases in order for the above to make sense. $\mathbf{v}$ can not be expressed in basis $\color{red}(\tilde{v}_1,\dots,\tilde{v}_n)$ if $\mathbf{T}$ only knows how to map basis vectors $\color{blue}({v}_1,\dots,{v}_n)$.
%
\[
T{\color{blue}v_j} = {\color{blue}T_{1,j}} w_1 + \cdots + {\color{blue}T_{m,j}} w_m
\]
%
%\[ Tv =  \sum_j \alpha_j T{\color{red}\tilde{v}_j} \]
}


\[
\underbrace{
\begin{pmatrix}
    T_{1,1} & \cdots & T_{1,n} \\
    \vdots & & \vdots\\
    T_{m,1} & \cdots & T_{m,n}
  \end{pmatrix}}_{\mathbf{T}}
  %
  \underbrace{
  \begin{pmatrix}
    c_1 \\
    \vdots \\
    c_n
  \end{pmatrix}}_{\mathbf{c}} =
  %
   \sum_{j=1}^n c_j \hspace{-0.6cm}
   \underbrace{\begin{pmatrix}
    {\color{red}T_{1,j}}  \\
    \vdots \\
    {\color{red}T_{m,j}}
  \end{pmatrix}}_{\mathrm{Tv_j~wrt~(w_1,\dots,w_m)}}
\]
%
\smallskip

Because recall that, for bases $v_1,\dots,v_n \in V$ and $w_1,\dots,w_m\in W$:
%
\[
Tv_j = {\color{red}T_{1,j}} w_1 + \cdots + {\color{red}T_{m,j}} w_m
\]

{
We see then that vector $c=\sum_j c_j v_j$ is mapped to $Tc = \sum_j c_j Tv_j$.

In other words, matrix product is behaving as expected.
}








