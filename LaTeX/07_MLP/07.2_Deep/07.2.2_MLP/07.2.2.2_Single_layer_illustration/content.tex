%!TEX root=../../../../root.tex


\begin{minipage}{\textwidth}
    \vspace{2em}
\end{minipage}

\begin{minipage}{.2\textwidth}
    \begin{overpic}
        [trim=0cm 0cm 0cm 0cm,clip,width=\linewidth]{07/layer1}
        \put(4,16){$\vdots$}
        \put(4,49){$\vdots$}
        \put(60,42){$\vdots$}
        %
        \put(-10,85){$x_1$}
        \put(-10,65){$x_2$}
        \put(-10,3){$x_n$}
        %
        \put(31,78){\color{blue}$w_{11}$}
        \put(31,60){\color{blue}$w_{12}$}
        \put(31,26){\color{blue}$w_{1n}$}
    \end{overpic}
    
    \vspace{2em}

    \begin{overpic}
        [trim=0cm 0cm 0cm 0cm,clip,width=\linewidth]{07/layer2}
        \put(4,16){$\vdots$}
        \put(4,49){$\vdots$}
        \put(60,42){$\vdots$}
        %
        \put(-10,85){$x_1$}
        \put(-10,65){$x_2$}
        \put(-10,3){$x_n$}
        %
        \put(31,60){\color{blue}$w_{m1}$}
        \put(31,34){\color{blue}$w_{m2}$}
        \put(31,10){\color{blue}$w_{mn}$}
    \end{overpic}%
    \vspace{2em}
\end{minipage}% This must go next to `\end{minipage}`
\begin{minipage}{.8\textwidth}
    \begin{equation}
        \sigma(\mathbf{Wx}) = \sigma \circ 
        \begin{pmatrix} 
            \color{blue} w_{11}  &  \color{blue}w_{12}  & \cdots & \color{blue}w_{1n} \\
             w_{21} 				 & w_{22} 				& \cdots & w_{2n}			   \\
            \vdots 				 & \cdots 				& \ddots & \vdots 			   \\
             w_{m1}             & w_{m2}	            &\cdots  & w_{mn}  
        \end{pmatrix} 
        \begin{pmatrix}
            x_1\\
            x_2\\
            \vdots\\
            x_n
        \end{pmatrix}
        = \sigma \circ 
        \begin{pmatrix}
            y_1\\
            y_2\\
            \vdots\\
            y_m
        \end{pmatrix}
    \end{equation}
    \vspace{4em}
    \begin{equation}
        \sigma(\mathbf{Wx}) = \sigma \circ 
        \begin{pmatrix} 
            w_{11}  &              w_{12}              & \cdots & w_{1n} \\
             w_{21} 				 & w_{22} 				& \cdots & w_{2n}			   \\
            \vdots 				 & \cdots 				& \ddots & \vdots 			   \\
            \color{blue} w_{m1}  & \color{blue}w_{m2}	&\cdots  & \color{blue}w_{mn}  
        \end{pmatrix} 
        \begin{pmatrix}
            x_1\\
            x_2\\
            \vdots\\
            x_n
        \end{pmatrix}
        = \sigma \circ 
        \begin{pmatrix}
            y_1\\
            y_2\\
            \vdots\\
            y_m
        \end{pmatrix}
    \end{equation}
    \vspace{2em}
\end{minipage}

To visualize the computation we usually take the hidden input representation $\vec{x}$ (possibly the input to the entire network), draw the corresponding $n$ nodes, then do the same for the output, drawing $m$ nodes.
Now, for every output node, we see which input nodes intervened to give raise to its value and draw the resulting edge. Each edge has a weight, and this is the corresponding element in the matrix.

\begin{center}
	\begin{overpic}
	[trim=0cm 0cm 0cm 0cm,clip,width=0.4\linewidth]{07/layer3}
	\put(4,16){$\vdots$}
	\put(4,49){$\vdots$}
	\put(60,42){$\vdots$}
	%
	\put(-10,85){$x_1$}
	\put(-10,65){$x_2$}
	\put(-10,3){$x_n$}
	%
	\put(60,71){$\sigma$}
	\end{overpic}%

\end{center}

For this model, the resulting network will be fully connected, but this will not be always the case.