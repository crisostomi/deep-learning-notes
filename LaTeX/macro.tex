%!TEX root=./root.tex

\newcommand{\resaltar}[1]{{\color{darkblue}#1}}
\newcommand{\redaltar}[1]{{\color{darkred}#1}}
\newcommand{\regaltar}[1]{{\color{darkgreen}#1}}

\def\Circlearrowleft{\ensuremath{%
  \rotatebox[origin=c]{180}{$\circlearrowleft$}}}
\def\Circlearrowright{\ensuremath{%
  \rotatebox[origin=c]{180}{$\circlearrowright$}}}
\def\CircleArrowleft{\ensuremath{%
  \reflectbox{\rotatebox[origin=c]{180}{$\circlearrowleft$}}}}
\def\CircleArrowright{\ensuremath{%
  \reflectbox{\rotatebox[origin=c]{180}{$\circlearrowright$}}}}

 \newcommand\shorteq{\mkern1.5mu{=}\mkern1.5mu}
 \newcommand\shorttimes{\mkern1.5mu{\times}\mkern1.5mu}

 
\newcommand*{\vertbar}{\rule[0.0ex]{0.5pt}{2.5ex}}
\newcommand*{\horzbar}{\rule[0.5ex]{2.5ex}{0.5pt}}

% % === CUSTOM TITLE
% \makeatletter
% \renewcommand*{\maketitle}{%
% \begin{titlepage}
%     \begin{center}
%         {\huge\bfseries\@title\unskip\strut\par}
%         \vspace{0.5cm}
%         {\Large\itshape\@author\unskip\strut\par}
%         \vspace{0.5cm}
%         {\large\@date\par}
%         \vfill
%         {\includegraphics[width=0.5\linewidth]{figures/sapienza_logo.png}\par}
%         \vfill
%         {\Large\itshape{A.Y. 2019/2020}\unskip\strut\par}
%     \end{center}
% \end{titlepage}
% }
% \makeatother

\newcommand{\inner}[2]{
    \langle #1, #2 \rangle
}

\newcommand{\partfrac}[2]{\frac{\partial #1}{\partial #2}}

% variable for the counter to make it work both for chapters and sections
\newcommand{\counterVariable}{chapter}
\makeatletter
\@ifundefined{chapter}{%
    \renewcommand{\counterVariable}{section}
}
\makeatother


% theorems etc
\newmdtheoremenv{unboxedthm}{Theorem}[\counterVariable]
\newenvironment{thm}{
    \begin{unboxedthm}
}
{
    \end{unboxedthm}
}

\newtheorem{lem}{Lemma}[\counterVariable]
\newtheorem{claim}{Claim}[\counterVariable]
\newtheorem{cor}{Corollary}[\counterVariable]
\newtheorem{prop}{Proposition}[\counterVariable]
\newtheorem{prob}{Problem}[\counterVariable]
\newtheorem{track}{Track}[\counterVariable]
\newtheorem*{track*}{Track}
\theoremstyle{definition}
\newtheorem{defn}{Definition}[\counterVariable]
% === EXAMPLES
%% set the counter for your environment
\newcounter{example}
\renewcommand{\theexample}{\the\counterVariable.\arabic{example}}
\counterwithin{example}{\counterVariable}

%% define the style
\mdfdefinestyle{example}{%
    backgroundcolor=gray!10,
    linecolor=gray,
    outerlinewidth=1pt,
    roundcorner=1mm,
    skipabove=\baselineskip,
    skipbelow=\baselineskip,
    frametitle=\mbox{},
}

%% setup the environments
%%% with number
\newmdenv[%
    style=example,
    settings={\global\refstepcounter{example}},
    frametitlefont={\bfseries Example~\theexample\quad},
]{example}
%%% without number (starred version)
\newmdenv[%
    style=example,
    frametitlefont={\bfseries Example~\quad},
]{example*}

% === PROBLEMS
%% set the counter for your environment
\newcounter{problem}
\renewcommand{\theproblem}{\the\counterVariable.\arabic{problem}}
\counterwithin{problem}{\counterVariable}

%% define the style
\mdfdefinestyle{problem}{%
    backgroundcolor=red!10,
    linecolor=gray,
    outerlinewidth=1pt,
    roundcorner=1mm,
    skipabove=\baselineskip,
    skipbelow=\baselineskip,
    frametitle=\mbox{},
}

%% setup the environments
%%% with number
\newmdenv[%
    style=problem,
    settings={\global\refstepcounter{problem}},
    frametitlefont={\bfseries Problem~\theproblem\quad},
]{problem}
%%% without number (starred version)
\newmdenv[%
    style=problem,
    frametitlefont={\bfseries Problem~\quad},
]{problem*}

\theoremstyle{remark}
\newtheorem{obs}{Observation}[\counterVariable]

\crefname{thm}{Theorem}{Theorems}
\crefname{thm}{Theorem}{Theorems}
\crefname{track}{Exercise}{Exercises}
\crefname{unboxedthm}{Theorem}{Theorems}
\crefname{claim}{Claim}{Claims}
\crefname{lem}{Lemma}{Lemmas}
\crefname{ex}{Example}{Examples}
\crefname{obs}{Observation}{Observations}
\crefname{cor}{Corollary}{Corollaries}
\crefname{defn}{Definition}{Definitions}
\crefname{prop}{Proposition}{Propositions}
\crefname{prob}{Problem}{Problems}

\renewcommand\lstlistingname{Algorithm}
\renewcommand\lstlistlistingname{Algorithms}

% math operators
\DeclareMathOperator{\EX}{\mathbb{E}}% expected value
\DeclareMathOperator*{\argmin}{\operatorname{argmin}}
\DeclareMathOperator*{\argmax}{\operatorname{argmax}}
\DeclareMathOperator*{\avg}{avg}
\usepackage{amsfonts}
\DeclareMathSymbol{\shortminus}{\mathbin}{AMSa}{"39}