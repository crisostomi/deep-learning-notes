%!TEX root = ../../root.tex

In the update law in \cref{eq:chap6:intro:update-law}, we have not yet clarified the role of $\alpha$. This is a \emph{hyperparameter}, meaning a parameter that does not belong to the model and does not influence the model directly, but does so indirectly by affecting the way it is fitted to the data. This parameter is always positive (otherwise we would maximize the loss function, by moving in the direction of the gradient) and is called \emph{learning rate}. 

The name is due to the fact that it influences the length of a ``learning'' step of the gradient descent algorithm, and hence the rate at which the algorithm makes progress. It is not precisely the step length itself, since that would be
\begin{equation}
    \norm{\mathbf{x}^{(t+1)} - \mathbf{x}^{(t)}} = \norm{\alpha \nabla f(\mathbf{x}^{(t)})} = \alpha \norm{\nabla f(\mathbf{x}^{(t)})}.
\end{equation}

Nonetheless, the direct proportionality that stands means that the value of this parameter can heavily influence the evolution of the algorithm. If this value is too small, then we have slow convergence speed, \textit{i.e.} we will need more steps to reach a minimum. If it is too large, we risk \emph{overshooting}, \textit{i.e.} the algorithm starts oscillating since even though the minimum is indeed in the direction of the gradient, reaching it would require a smaller step and therefore the algorithm keeps going back and forth around it, something that is often referred to as \emph{divergence} (or \emph{convergence issues}).

\begin{figure}[H]
    \centering
    \begin{overpic}
    [trim=0cm 0cm 0cm 0cm,clip,width=0.8\linewidth]{06/steps3}
            \put(11,-3){\footnotesize small $\alpha$}
                    \put(45,-3){\footnotesize large $\alpha$}
            \put(77,-3){\footnotesize optimal $\alpha$}
            \put(63,-6){\footnotesize $\arg\min_\alpha f( \mathbf{x}^{(t)}  - \alpha\nabla f( \mathbf{x}^{(t)} ))$}
    \end{overpic}
    \vspace{2em}
    \caption{Comparison of values of the learning rate $\alpha$.}
\end{figure}

Since we move in straight lines, and the function is nonlinear, we cannot know what is the overall best value for $\alpha$; if this was possible, we could just make one step with the optimal $\alpha$, i.e. a single computation, a closed-form solution, but as we know this is not (guaranteed to be) possible in the nonlinear setting. Indeed, recall that the gradient is only a local linear approximation of the behavior of the function, so we cannot hope to have just one step length that, when executed along a straight line, works well everywhere. 

Instead, we can locally adjust the value of $\alpha$ according to the local information we have at a certain point, with a technique called \emph{line search}. What it does is very simple: once the direction of steepest descent is known ($- \nabla f( \mathbf{x}^{(t)})$), we seek the value $\alpha$ such that taking a step of size $\alpha \norm{\nabla f( \mathbf{x}^{(t)} )}$ in that direction makes the function decrease the most. So we have an optimization procedure inside an optimization algorithm.

\subsection{Decay} 
%!TEX root = ../root.tex

\section{Entropy}

\paragraph{Intuition}

In information theory, the \emph{entropy} of a random variable is the average level of ``information'', or dually ``uncertainty'', that is inherent in the variable's possible outcomes.

Let's look at one such outcome, let's call it the event $E$, and say that it will occur with probability $p(E)$.

The intuition is that if the event $E$ is very likely to happen, then knowing that it will happen does not bring any interesting information. On the contrary, what is truly informative is knowing that something that happens very infrequently will indeed happen.

Therefore, the \emph{information content} (or \emph{surprisal}) carried by knowing that the event $E$ will happen can be quantified as a function that decreases with $p(E)$. In particular, this function is defined as:
\begin{equation}
	I(E) \triangleq -\log p(E) = \log \frac{1}{p(E)}.
\end{equation}

A random variable has a probability distribution defined over all the events that it encodes. Then, we can compute the average information inherent in the variable as the weighted average of the information carried by each one of its events, weighted by the probability that it will actually happen. 

A random variable in which all the outcomes are equally likely has high entropy, since there is maximum uncertainty about its outcome. A random variable in which there are certain outcomes that are much more likely than others has low entropy, since there is less uncertainty about its outcome.

A random variable that has \emph{more} outcomes than another has \emph{higher} entropy, since there is more uncertainty about its outcome.

\paragraph{Definition}

Let $X$ be a random variable, with possible values $\{ x_1, \dots, x_n \}$. Let $P(X)$ be the probability mass function defining a probability distribution over all the possible values of $X$.

Then, we call \emph{entropy} the average information represented in the distribution:
\begin{equation}
	H(X) \triangleq - \sum_{i=1}^{n} p(x_i) \log p(x_i),
\end{equation}
where $p(x_i) \equiv P(X = x_i)$.

\section{Kullback-Leibler divergence}

\paragraph{Intuition}

Let's consider two distributions of probability $p$ and $q$. Usually, $p$ represents the data, while $q$ a model, or in general an approximation of $p$, and we want to know how good of an approximation this is.

Then the Kullback-Leibler divergence is interpreted as the average loss of information content that we have when representing samples of $p$ (the data) using an \emph{optimal code} (something that does not introduce additional uncertainty beside the one intrinsic to the distribution) for $q$ (the model) instead of an optimal code for $p$.

\paragraph{Definition}
Let $p$ and $q$ be two probability distributions defined over the same space $X$. 

Then we call Kullback-Leibler divergence the measure
\begin{equation}
    KL (p \| q) \triangleq \sum_{x \in X} p(x) \log \left( \frac{p(x)}{q(x)} \right)
\end{equation}
which is equivalent to
\begin{equation}
    KL (p \| q) = -\sum_{x \in X} p(x) \log \left( \frac{q(x)}{p(x)} \right).
\end{equation}

\paragraph{Properties}

The KL-divergence:
\begin{itemize}
    \item is \emph{not} symmetric.
    \item is \emph{not} a distance, since it is not symmetric.
    \item is always non-negative.
    \item can be expressed as an expectation.
    \begin{equation}
        \begin{aligned}
            KL (P \| Q) & = \sum_{x \in X} p(x) \log \left( \frac{p(x)}{q(x)} \right) \\
            & = \mathbb{E}_{x \sim p(x)} \log \left( \frac{p(x)}{q(x)} \right).
        \end{aligned}
    \end{equation}
    \item measures the dissimilarity of two distributions in terms of their entropy.
    \begin{equation}\label{eq:KL-divergence}
        KL(p \| q) \approx H(q) - H(p)
    \end{equation} 
    
    In fact, Substituting the definition of entropy in \cref{eq:KL-divergence} we go back to the definition.
    \begin{align}
        KL(p \| q) &\approx - \sum_{x}q(x)\log \ q(x) + \sum_{x}p(x)\log \ p(x) \\
        KL(p \| q) &= - \sum_{x}p(x)\log \ q(x) + \sum_{x}p(x)\log \ p(x) \\
        KL(p \| q) &= \sum_{x}p(x)\log \frac{p(x)}{q(x)}
    \end{align}
\end{itemize}

\subsection{Momentum} 
%!TEX root = ../root.tex

\section{Entropy}

\paragraph{Intuition}

In information theory, the \emph{entropy} of a random variable is the average level of ``information'', or dually ``uncertainty'', that is inherent in the variable's possible outcomes.

Let's look at one such outcome, let's call it the event $E$, and say that it will occur with probability $p(E)$.

The intuition is that if the event $E$ is very likely to happen, then knowing that it will happen does not bring any interesting information. On the contrary, what is truly informative is knowing that something that happens very infrequently will indeed happen.

Therefore, the \emph{information content} (or \emph{surprisal}) carried by knowing that the event $E$ will happen can be quantified as a function that decreases with $p(E)$. In particular, this function is defined as:
\begin{equation}
	I(E) \triangleq -\log p(E) = \log \frac{1}{p(E)}.
\end{equation}

A random variable has a probability distribution defined over all the events that it encodes. Then, we can compute the average information inherent in the variable as the weighted average of the information carried by each one of its events, weighted by the probability that it will actually happen. 

A random variable in which all the outcomes are equally likely has high entropy, since there is maximum uncertainty about its outcome. A random variable in which there are certain outcomes that are much more likely than others has low entropy, since there is less uncertainty about its outcome.

A random variable that has \emph{more} outcomes than another has \emph{higher} entropy, since there is more uncertainty about its outcome.

\paragraph{Definition}

Let $X$ be a random variable, with possible values $\{ x_1, \dots, x_n \}$. Let $P(X)$ be the probability mass function defining a probability distribution over all the possible values of $X$.

Then, we call \emph{entropy} the average information represented in the distribution:
\begin{equation}
	H(X) \triangleq - \sum_{i=1}^{n} p(x_i) \log p(x_i),
\end{equation}
where $p(x_i) \equiv P(X = x_i)$.

\section{Kullback-Leibler divergence}

\paragraph{Intuition}

Let's consider two distributions of probability $p$ and $q$. Usually, $p$ represents the data, while $q$ a model, or in general an approximation of $p$, and we want to know how good of an approximation this is.

Then the Kullback-Leibler divergence is interpreted as the average loss of information content that we have when representing samples of $p$ (the data) using an \emph{optimal code} (something that does not introduce additional uncertainty beside the one intrinsic to the distribution) for $q$ (the model) instead of an optimal code for $p$.

\paragraph{Definition}
Let $p$ and $q$ be two probability distributions defined over the same space $X$. 

Then we call Kullback-Leibler divergence the measure
\begin{equation}
    KL (p \| q) \triangleq \sum_{x \in X} p(x) \log \left( \frac{p(x)}{q(x)} \right)
\end{equation}
which is equivalent to
\begin{equation}
    KL (p \| q) = -\sum_{x \in X} p(x) \log \left( \frac{q(x)}{p(x)} \right).
\end{equation}

\paragraph{Properties}

The KL-divergence:
\begin{itemize}
    \item is \emph{not} symmetric.
    \item is \emph{not} a distance, since it is not symmetric.
    \item is always non-negative.
    \item can be expressed as an expectation.
    \begin{equation}
        \begin{aligned}
            KL (P \| Q) & = \sum_{x \in X} p(x) \log \left( \frac{p(x)}{q(x)} \right) \\
            & = \mathbb{E}_{x \sim p(x)} \log \left( \frac{p(x)}{q(x)} \right).
        \end{aligned}
    \end{equation}
    \item measures the dissimilarity of two distributions in terms of their entropy.
    \begin{equation}\label{eq:KL-divergence}
        KL(p \| q) \approx H(q) - H(p)
    \end{equation} 
    
    In fact, Substituting the definition of entropy in \cref{eq:KL-divergence} we go back to the definition.
    \begin{align}
        KL(p \| q) &\approx - \sum_{x}q(x)\log \ q(x) + \sum_{x}p(x)\log \ p(x) \\
        KL(p \| q) &= - \sum_{x}p(x)\log \ q(x) + \sum_{x}p(x)\log \ p(x) \\
        KL(p \| q) &= \sum_{x}p(x)\log \frac{p(x)}{q(x)}
    \end{align}
\end{itemize}


