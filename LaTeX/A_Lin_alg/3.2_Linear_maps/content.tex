%!TEX root=../../root.tex

A \emph{linear map} from $V$ to $W$ is a function $T:V\to W$ with the properties:
\begin{itemize}
\item \textbf{additivity}: $T(u+v) = Tu + Tv$ for all $u,v\in V$
\item \textbf{homogeneity}: $T(\lambda v) = \lambda (Tv)$ for all $\lambda \in\mathbb{R}$ and all $v\in V$
\end{itemize}

\textbf{Examples:}

\begin{itemize}
\item  identity $I:V\to V$, defined as $Iv = v$ 
\item differentiation $D: \mathcal{F}(\mathbb{R})\to  \mathcal{F}(\mathbb{R})$, defined as $Df = f'$
\item integration $T: \mathcal{F}(\mathbb{R}) \to \mathbb{R}$, defined as $Tf = \int_0^1 f(x) dx$
\item from $\mathbb{R}^3$ to $\mathbb{R}^2$, defined as \[T(x,y,z) = (2x-y+3z, 7x+5y-6z)\]
\item from $\mathbb{R}^n$ to $\mathbb{R}^m$, defined as \[T(x_1,\dots,x_n) = (A_{1,1}x_1 + \cdots + A_{1,n}x_n, \dots, A_{m,1}x_1 + \cdots + A_{m,n} x_n)\]
\item equation of a line
\[
y = ax + b
\]
\item equations such as the following, note that this is linear with respect to $sin(x)$
%
\[
y = z\sin(x) + z^2\sin(x)
\]
\item \emph{Reflection} operation on an image:
%
\[
T : \mathbb{R}^2 \to \mathbb{R}^2\,,\quad T(x,y) = (-x,y)
\]
\end{itemize}

\begin{center}
		\begin{overpic}
		[trim=0cm 0cm 0cm 0cm,clip,width=0.25\linewidth]{03/molecule}
		\put(113,33){$\Rightarrow$}
		\end{overpic}\hspace{1cm}
		\begin{overpic}
		[trim=0cm 0cm 0cm 0cm,clip,width=0.25\linewidth]{03/molecule_lr}
		\end{overpic}
\end{center}

Linear maps $T:V\to W$ form a \emph{vector space}, with addition and multiplication defined as:
%
\begin{align*}
(S+T)(v) &= Sv + Tv\\
(\lambda T)(v) &= \lambda (Tv)
\end{align*}
%

{
In addition, we also have a useful definition of \emph{product} between linear maps. This is kind of a special situation, since multiplying vectors doesn't necessarily make sense in general. 
}

{
If $T:U\to {\color{red}V}$ and $S:{\color{red}V} \to W$, their product $ST:U \to W$ is defined by
%
\[
(ST)(u) = S(Tu)
\]
%
In other words, $ST$ is just the usual composition $S\circ T$ of two functions
}


\begin{itemize}
\item \textbf{associativity}: $(T_1 T_2)T_3 = T_1 (T_2 T_3)$

\item \textbf{identity}: $TI = IT = T$

\item \textbf{distributive properties}: $(S_1+S_2)T = S_1 T + S_2 T$ and $S(T_1 + T_2) = S T_1 + ST_2$
\end{itemize}

{
Keep in mind that composition of linear maps \emph{is not commutative}, i.e. \[ST\neq TS\] in general (although there are special cases).

\textbf{Example:} Take $Sf = f'$ and $(Tf)(x) = x^2 f(x)$.
}

\subsection{Matrices} 
%!TEX root=../../../root.tex

Consider a linear map $T:V \to W$, a basis $v_1,\dots,v_n \in V$ and a basis $w_1,\dots,w_m\in W$.
{
The \emph{matrix} of $T$ in these bases is the $m\times n$ array of values in $\mathbb{R}$
%
\[
\mathbf{T} = \begin{pmatrix}
    T_{1,1} & \cdots & T_{1,n} \\
    \vdots & & \vdots\\
    T_{m,1} & \cdots & T_{m,n}
  \end{pmatrix}
\]
%
whose entries $T_{i,j}$ are defined by
%
\[
{\color{darkgreen}Tv_j} = {\color{red}T_{1,j}} w_1 + \cdots + {\color{red}T_{m,j}} w_m
\]
}%

{
Hence each column of $\mathbf{T}$ contains the {\color{red}linear combination coefficients} for the {\color{darkgreen}image via $T$ of a basis vector from $V$}
}

{
In other words, the matrix encodes {\color{darkgreen}how basis vectors are mapped}, and this is enough to map all other vectors in their span, since:
\[ Tv = T ( \sum_j \alpha_j v_j ) = \sum_j T(\alpha_j v_j) = \sum_j \alpha_j {\color{darkgreen}Tv_j} \]
}

The matrix is a \emph{representation} for a linear map, and it \emph{depends on the choice of bases}.
Suppose $v \in V$ is an arbitrary vector, while $v_1,\dots,v_n$ is a basis of $V$. The matrix of $v$ wrt this basis is the $n\times 1$ matrix:
%
\[
\mathbf{v} = \begin{pmatrix}
    c_1 \\
    \vdots\\
    c_n
  \end{pmatrix}
\]
%
so that
%
\[
v = c_1 v_1 + \cdots c_n v_n
\]

Once again, we see that the matrix \emph{depends on the choice of basis} for $V$


\begin{itemize}
\item \textbf{addition}:  the matrix of $S+T$ can be obtained by summing the matrices of $S$ and $T${; this only makes sense if the \emph{same bases} are used for $S$, $T$, and $S+T$}
\item \textbf{scalar multiplication}: given $\lambda\in\mathbb{R}$, the matrix for $\lambda T$ is given by $\lambda$ times the matrix of $T$
\end{itemize}

{
In fact, we have just shown that \emph{matrices form a vector space} (Q1: what is the additive identity?) {(Q2: what is the vector space dimension?)}
}

{
We call $\mathbb{R}^{m\times n}$ the vector space of all $m\times n$ matrices with values in $\mathbb{R}$
}

{
\begin{itemize}
\item \textbf{product}: the matrix for $ST$ can be computed by the \emph{matrix product} between $\mathbf{S}$ and $\mathbf{T}$; in fact, the matrix product is defined precisely to make this work

{
Q3: is matrix product commutative?
}

{
Q4: do we need the same bases for $S:U\to V$ and $T:V \to W$?
}

\end{itemize}
}



Consider a linear map $T:V \to W$, a basis $v_1,\dots,v_n \in V$ and a basis $w_1,\dots,w_m\in W$.

%\medskip
%The $k$-th column of $\mathbf{T}$ equals the matrix vector $\mathbf{v}_k$:

%figure

From the definition of matrix product, one can show that it operates on a vector matrix as expected:
\[
\mathbf{Tv} =\mathbf{w} \quad \Leftrightarrow \quad Tv=w
\]
where $\mathbf{Tv}$ is the matrix product of $\mathbf{T}$ and $\mathbf{v}$, while $Tv$ simply denotes the function evaluation $T(v)$


{
\textbf{Remember:} $\mathbf{T}, \mathbf{v}, \mathbf{w}$ must follow a coherent choice of bases in order for the above to make sense. $\mathbf{v}$ can not be expressed in basis $\color{red}(\tilde{v}_1,\dots,\tilde{v}_n)$ if $\mathbf{T}$ only knows how to map basis vectors $\color{blue}({v}_1,\dots,{v}_n)$.
%
\[
T{\color{blue}v_j} = {\color{blue}T_{1,j}} w_1 + \cdots + {\color{blue}T_{m,j}} w_m
\]
%
%\[ Tv =  \sum_j \alpha_j T{\color{red}\tilde{v}_j} \]
}


\[
\underbrace{
\begin{pmatrix}
    T_{1,1} & \cdots & T_{1,n} \\
    \vdots & & \vdots\\
    T_{m,1} & \cdots & T_{m,n}
  \end{pmatrix}}_{\mathbf{T}}
  %
  \underbrace{
  \begin{pmatrix}
    c_1 \\
    \vdots \\
    c_n
  \end{pmatrix}}_{\mathbf{c}} =
  %
   \sum_{j=1}^n c_j \hspace{-0.6cm}
   \underbrace{\begin{pmatrix}
    {\color{red}T_{1,j}}  \\
    \vdots \\
    {\color{red}T_{m,j}}
  \end{pmatrix}}_{\mathrm{Tv_j~wrt~(w_1,\dots,w_m)}}
\]
%
\smallskip

Because recall that, for bases $v_1,\dots,v_n \in V$ and $w_1,\dots,w_m\in W$:
%
\[
Tv_j = {\color{red}T_{1,j}} w_1 + \cdots + {\color{red}T_{m,j}} w_m
\]

{
We see then that vector $c=\sum_j c_j v_j$ is mapped to $Tc = \sum_j c_j Tv_j$.

In other words, matrix product is behaving as expected.
}








