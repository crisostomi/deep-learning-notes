%!TEX root=../../root.tex

The motivation for the definition of a vector space comes from the classical properties of addition and scalar multiplication.
A \emph{vector space} $V$ over a field $F$ is a set equipped with two operations, $+: V \to V$ and $\cdot: F \times V \to V$, often referred to as \emph{addition} and \emph{scalar multiplication} respectively, that satisfy the following properties:

\begin{itemize}
\item \textbf{commutativity}: $u+v = v+u$ for all $u,v\in V$;

\item \textbf{associativity}: $(u+v)+w = u+(v+w)$ and $(ab)v=a(bv)$ for all $u,v,w\in V$ and all $a,b\in\mathbb{R}$; 
\item the set is closed wrt to the two operations, so $u+v\in V$ and $av\in V$: ``what happens in Vegas, stays in Vegas'';


\item \textbf{additive identity}: there exists an element $0\in V$ such that $v+0=v$ for all $v\in V$

\item \textbf{additive inverse}: for every $v\in V$, there exists $w\in V$ such that $v+w=0$

\item \textbf{multiplicative identity}: $1v = v$ for all $v\in V$

\item \textbf{distributive properties}: $a(u+v)=au+av$ and $(a+b)v=av+bv$ for all $a,b\in\mathbb{R}$ and all $u,v\in V$
\end{itemize}

$\mathbb{R}^n$ is defined as the set of all $n$-long sequences of numbers in $\mathbb{R}$:

\[
\mathbb{R}^n = \{ (x_1,\dots,x_n) : x_j \in \mathbb{R} ~\mathrm{for}~ j=1,2,\dots,n\}
\]

{
\medskip
Addition and scalar multiplication are defined as expected:
%
\begin{align*}
(x_1,x_2,\dots,x_n) + (y_1,y_2,\dots,y_n) &= (x_1+y_1,x_2+y_2,\dots,x_n+y_n)\\
\lambda (x_1, x_2,\dots,x_n) &= (\lambda x_1, \lambda x_2, \dots, \lambda x_n)
\end{align*}
}

While the additive identity can be defined as:
\[ 0 = (0, \dots , 0) \]
With these definitions, $\mathbb{R}^n$ is a vector space, usually defined over the scalar field $\mathbb{R}$.

Consider the set of all functions $f: [0,1] \to \mathbb{R}$ with the standard definitions for sum and scalar product:
%
\begin{align*}
(f+g)(x) &= f(x) + g(x)\\
(\lambda f)(x) &= \lambda f(x)
\end{align*}
%
for all $x\in[0,1]$ and $\lambda \in\mathbb{R}$, with additive identity and inverse defined as:
%
\begin{align*}
0(x) &= 0\\
(-f)(x) &= -f(x)
\end{align*}
%
for all $x\in[0,1]$. The above forms a vector space. In fact, \emph{any} set of functions $f: S\to\mathbb{R}$ with $S\neq\emptyset$ (Q: why?) and the definitions above forms a vector space.

Elements of a vector space (called \emph{vectors}) are not necessarily lists. A vector space is an \emph{abstract} entity whose elements might be lists, functions, or weird objects. Surfaces do not form a vector space, as the sum of two points over a surface is not defined. Surfaces can be studied using \emph{differential geometry}, which is a mathematical discipline that uses the techniques of differential calculus, integral calculus, linear algebra and multilinear algebra to study problems in geometry; we'll need it for studying the \emph{manifold hypothesis} and \emph{geometric deep learning}.

\begin{figure}[H]
\begin{center}
{
		\begin{overpic}
		[trim=0cm 0cm 0cm 0cm,clip,width=0.4\linewidth]{03/bunny.png}
		\end{overpic}
}%
{
		\begin{overpic}
		[trim=0cm 0cm 0cm 0cm,clip,width=0.4\linewidth]{03/bunnyfun.png}
		\end{overpic}
}
\end{center}
\caption{An example of surface.}
\end{figure}
{
\medskip


\medskip
We can still use linear algebra to manipulate \emph{functions on surfaces}.
}

A subset $U\subset V$ is a \emph{subspace} of $V$ if it is a vector space (using the same operations defined for $V$). In particular:

\begin{itemize}
	\item $0\in U$
	\item $u,v\in U$ implies $u+v\in U$
	\item $u\in U$ implies $\alpha u \in U$ for any $\alpha\in\mathbb{R}$
\end{itemize}


\textbf{Examples:}
\medskip

\begin{itemize}
	\item $\{ (x_1, x_2, 0) : x_1, x_2 \in \mathbb{R}\}$ is a subspace of $\mathbb{R}^3$
	\item The set of \emph{piecewise-linear functions} on a graph $G=(V,E)$ is a subspace of all functions $f:V\to\mathbb{R}$
\end{itemize}

\subsection{Basis} 
%!TEX root=../../../root.tex

Consider a linear map $T:V \to W$, a basis $v_1,\dots,v_n \in V$ and a basis $w_1,\dots,w_m\in W$.
{
The \emph{matrix} of $T$ in these bases is the $m\times n$ array of values in $\mathbb{R}$
%
\[
\mathbf{T} = \begin{pmatrix}
    T_{1,1} & \cdots & T_{1,n} \\
    \vdots & & \vdots\\
    T_{m,1} & \cdots & T_{m,n}
  \end{pmatrix}
\]
%
whose entries $T_{i,j}$ are defined by
%
\[
{\color{darkgreen}Tv_j} = {\color{red}T_{1,j}} w_1 + \cdots + {\color{red}T_{m,j}} w_m
\]
}%

{
Hence each column of $\mathbf{T}$ contains the {\color{red}linear combination coefficients} for the {\color{darkgreen}image via $T$ of a basis vector from $V$}
}

{
In other words, the matrix encodes {\color{darkgreen}how basis vectors are mapped}, and this is enough to map all other vectors in their span, since:
\[ Tv = T ( \sum_j \alpha_j v_j ) = \sum_j T(\alpha_j v_j) = \sum_j \alpha_j {\color{darkgreen}Tv_j} \]
}

The matrix is a \emph{representation} for a linear map, and it \emph{depends on the choice of bases}.
Suppose $v \in V$ is an arbitrary vector, while $v_1,\dots,v_n$ is a basis of $V$. The matrix of $v$ wrt this basis is the $n\times 1$ matrix:
%
\[
\mathbf{v} = \begin{pmatrix}
    c_1 \\
    \vdots\\
    c_n
  \end{pmatrix}
\]
%
so that
%
\[
v = c_1 v_1 + \cdots c_n v_n
\]

Once again, we see that the matrix \emph{depends on the choice of basis} for $V$


\begin{itemize}
\item \textbf{addition}:  the matrix of $S+T$ can be obtained by summing the matrices of $S$ and $T${; this only makes sense if the \emph{same bases} are used for $S$, $T$, and $S+T$}
\item \textbf{scalar multiplication}: given $\lambda\in\mathbb{R}$, the matrix for $\lambda T$ is given by $\lambda$ times the matrix of $T$
\end{itemize}

{
In fact, we have just shown that \emph{matrices form a vector space} (Q1: what is the additive identity?) {(Q2: what is the vector space dimension?)}
}

{
We call $\mathbb{R}^{m\times n}$ the vector space of all $m\times n$ matrices with values in $\mathbb{R}$
}

{
\begin{itemize}
\item \textbf{product}: the matrix for $ST$ can be computed by the \emph{matrix product} between $\mathbf{S}$ and $\mathbf{T}$; in fact, the matrix product is defined precisely to make this work

{
Q3: is matrix product commutative?
}

{
Q4: do we need the same bases for $S:U\to V$ and $T:V \to W$?
}

\end{itemize}
}



Consider a linear map $T:V \to W$, a basis $v_1,\dots,v_n \in V$ and a basis $w_1,\dots,w_m\in W$.

%\medskip
%The $k$-th column of $\mathbf{T}$ equals the matrix vector $\mathbf{v}_k$:

%figure

From the definition of matrix product, one can show that it operates on a vector matrix as expected:
\[
\mathbf{Tv} =\mathbf{w} \quad \Leftrightarrow \quad Tv=w
\]
where $\mathbf{Tv}$ is the matrix product of $\mathbf{T}$ and $\mathbf{v}$, while $Tv$ simply denotes the function evaluation $T(v)$


{
\textbf{Remember:} $\mathbf{T}, \mathbf{v}, \mathbf{w}$ must follow a coherent choice of bases in order for the above to make sense. $\mathbf{v}$ can not be expressed in basis $\color{red}(\tilde{v}_1,\dots,\tilde{v}_n)$ if $\mathbf{T}$ only knows how to map basis vectors $\color{blue}({v}_1,\dots,{v}_n)$.
%
\[
T{\color{blue}v_j} = {\color{blue}T_{1,j}} w_1 + \cdots + {\color{blue}T_{m,j}} w_m
\]
%
%\[ Tv =  \sum_j \alpha_j T{\color{red}\tilde{v}_j} \]
}


\[
\underbrace{
\begin{pmatrix}
    T_{1,1} & \cdots & T_{1,n} \\
    \vdots & & \vdots\\
    T_{m,1} & \cdots & T_{m,n}
  \end{pmatrix}}_{\mathbf{T}}
  %
  \underbrace{
  \begin{pmatrix}
    c_1 \\
    \vdots \\
    c_n
  \end{pmatrix}}_{\mathbf{c}} =
  %
   \sum_{j=1}^n c_j \hspace{-0.6cm}
   \underbrace{\begin{pmatrix}
    {\color{red}T_{1,j}}  \\
    \vdots \\
    {\color{red}T_{m,j}}
  \end{pmatrix}}_{\mathrm{Tv_j~wrt~(w_1,\dots,w_m)}}
\]
%
\smallskip

Because recall that, for bases $v_1,\dots,v_n \in V$ and $w_1,\dots,w_m\in W$:
%
\[
Tv_j = {\color{red}T_{1,j}} w_1 + \cdots + {\color{red}T_{m,j}} w_m
\]

{
We see then that vector $c=\sum_j c_j v_j$ is mapped to $Tc = \sum_j c_j Tv_j$.

In other words, matrix product is behaving as expected.
}








